\documentclass[conference]{IEEEtran}
\IEEEoverridecommandlockouts
\usepackage{cite}
\usepackage{float}
\usepackage{amsmath,amssymb,amsfonts}
\usepackage{algorithmic}
\usepackage{graphicx}
\usepackage{textcomp}
\usepackage{xcolor}
\usepackage{alphabeta}
\usepackage{array}
\usepackage{tabularx}
\usepackage{listings}
\usepackage{hyperref}

\def\BibTeX{{\rm B\kern-.05em{\sc i\kern-.025em b}\kern-.08em
    T\kern-.1667em\lower.7ex\hbox{E}\kern-.125emX}}
\begin{document}

\title{Diseño y Análisis de Mecanismo Caminante Tipo Theo Jansen:\\Aplicación de Cinemática y Cinética de Sistemas Articulados\\}

\author{
\IEEEauthorblockN{1\textsuperscript{ro} Sebastián Andrés Rodríguez Carrillo}
\IEEEauthorblockA{\textit{Universidad Militar de Nueva Granada} \\
\textit{Ingeniería Mecatrónica}\\
est.sebastian.arod2@unimilitar.edu.co}
\and
\IEEEauthorblockN{2\textsuperscript{do} David Andrés Rodríguez Rozo}
\IEEEauthorblockA{\textit{Universidad Militar de Nueva Granada} \\
\textit{Ingeniería Mecatrónica}\\
est.david.arodrigu1@unimilitar.edu.co}
\and
\IEEEauthorblockN{3\textsuperscript{ro} Daniel García Araque}
\IEEEauthorblockA{\textit{Universidad Militar de Nueva Granada} \\
\textit{Ingeniería Mecatrónica}\\
est.daniel.garciaa@unimilitar.edu.co}
}

\maketitle

\begin{abstract}
Este informe presenta el diseño, análisis y construcción de un mecanismo caminante inspirado en los diseños de Theo Jansen para el curso de Dinámica Aplicada. Se aplicó la metodología de análisis cinemático mediante el método de circuitos vectoriales y análisis cinético mediante ecuaciones de Newton-Euler para sistemas multicuerpo. El mecanismo de 8 barras fue diseñado en SolidWorks y simulado en MATLAB para calcular posiciones, velocidades, aceleraciones, fuerzas y torques en las articulaciones. Se construyó un prototipo funcional con materiales limitados (MDF, acrílico, PLA) dentro de las restricciones dimensionales de 40×30×20 cm y masa máxima de 1.5 kg. Las pruebas experimentales validaron el movimiento caminante estable con contacto con el suelo superior al 75\% del ciclo de paso. Los resultados mostraron concordancia entre las predicciones teóricas y el comportamiento real del mecanismo, logrando una velocidad promedio de [XX] cm/s y estabilidad de [YY] apoyos simultáneos. El prototipo demostró capacidad de transporte y movimiento fluido sin volcamiento, cumpliendo los objetivos establecidos para la competencia.
\end{abstract}

\vspace{0.1cm}

\begin{IEEEkeywords}
Mecanismo Theo Jansen, cinemática de mecanismos, análisis cinético, sistemas articulados, método de circuitos vectoriales, Newton-Euler, SolidWorks, MATLAB, locomoción caminante, mecanismo de 8 barras
\end{IEEEkeywords}


\section{Introducción}

El diseño de mecanismos caminantes ha sido objeto de estudio en la ingeniería mecánica desde hace décadas, particularmente en aplicaciones de robótica móvil y vehículos todo terreno. Los mecanismos articulados ofrecen ventajas sobre sistemas de ruedas en terrenos irregulares, permitiendo superar obstáculos y adaptarse a superficies complejas \cite{todd1985walking}.

El presente proyecto aborda el diseño y construcción de un mecanismo caminante inspirado en las creaciones del artista cinético Theo Jansen, quien desarrolló criaturas mecánicas autopropulsadas que caminan mediante sistemas de eslabones articulados. El mecanismo de 8 barras de Jansen genera una trayectoria de paso característica que imita el movimiento de locomoción animal, con fases claramente definidas de apoyo y balanceo \cite{jansen2007great}.

El proyecto debe cumplir restricciones específicas de dimensiones (40×30×20 cm), masa máxima (1.5 kg), y operación mediante propulsión manual o motorizada de baja potencia (máximo 12V/2A). El mecanismo debe caminar de forma estable en línea recta sin perder contacto con el suelo en más del 25\% del ciclo de paso, operando sobre superficie plana en una pista de 1.5 metros.

La metodología de análisis se fundamenta en dos pilares: (1) Cinemática mediante el método de circuitos vectoriales para determinar posiciones, velocidades y aceleraciones de todos los puntos del mecanismo \cite{norton2011design}, y (2) Cinética mediante ecuaciones de Newton-Euler para calcular fuerzas de reacción en articulaciones y torques requeridos en el actuador \cite{uicker2003theory}.

En el aspecto de diseño, se utilizó modelado CAD tridimensional en SolidWorks para crear el ensamble completo del mecanismo, permitiendo verificar interferencias, calcular masas y momentos de inercia de cada eslabón. La simulación numérica en MATLAB complementa el análisis teórico mediante integración numérica de las ecuaciones de movimiento y generación de gráficas de trayectorias, velocidades y aceleraciones.

Este informe documenta el proceso completo desde la investigación de proporciones geométricas óptimas del mecanismo Jansen, modelado CAD, desarrollo de simulaciones cinemáticas y cinéticas, fabricación del prototipo, hasta las pruebas experimentales que validan el desempeño del sistema. Los resultados permiten comprender la aplicación práctica de los principios de dinámica de sistemas articulados en un proyecto de ingeniería mecatrónica.

\vspace{9pt}

\section{Objetivos}

\subsection{Objetivo General}

Diseñar, fabricar y analizar un mecanismo caminante tipo Theo Jansen aplicando los conceptos de cinemática y cinética de sistemas articulados, validando el desempeño mediante pruebas experimentales en competencia.

\subsection{Objetivos Específicos}

\begin{enumerate}
\item Analizar la geometría del mecanismo de 8 barras de Theo Jansen y seleccionar proporciones óptimas de eslabones para generar trayectoria de paso estable.

\item Desarrollar el modelo cinemático del mecanismo mediante el método de circuitos vectoriales, calculando posiciones, velocidades y aceleraciones de todos los puntos.

\item Realizar análisis cinético mediante ecuaciones de Newton-Euler para determinar fuerzas en articulaciones y torques requeridos en el actuador.

\item Diseñar el mecanismo completo en SolidWorks con materiales permitidos (MDF, acrílico, PLA) dentro de las restricciones dimensionales y de masa.

\item Implementar simulaciones numéricas en MATLAB para validar el comportamiento cinemático y cinético del mecanismo.

\item Fabricar el prototipo funcional con sistema de transmisión y verificar ensamblaje e interferencias.

\item Realizar pruebas experimentales midiendo velocidad, estabilidad y consumo energético, comparando con resultados teóricos.

\item Evaluar el desempeño del mecanismo en competencia según criterios establecidos (movimiento estable, velocidad, diseño técnico, análisis dinámico).
\end{enumerate}

\vspace{9pt}

\section{Marco Teórico}

\vspace{9pt}

\subsection{Mecanismos Tipo Theo Jansen}

\subsubsection{Historia y Principio de Funcionamiento}

Los mecanismos Theo Jansen son sistemas de eslabones articulados desarrollados por el artista cinético holandés Theo Jansen desde 1990. Estas estructuras, conocidas como "Strandbeests" (bestias de playa), son capaces de caminar impulsadas por el viento mediante un sistema de velas \cite{jansen2007great}.

El mecanismo fundamental consiste en un sistema de 8 barras (eslabones) conectados mediante articulaciones de revolución que transforman el movimiento rotacional continuo de una manivela en un movimiento de locomoción caminante. La geometría específica de las longitudes de los eslabones genera una trayectoria característica del punto de contacto con el suelo que imita el paso de animales cuadrúpedos.

\subsubsection{Proporciones Geométricas Clásicas}

Las proporciones originales de Theo Jansen fueron optimizadas mediante algoritmos genéticos para maximizar la longitud de paso y minimizar la variación vertical del centro de masa \cite{nansai2015evolutionary}. Las longitudes estándar de los eslabones son:

\begin{table}[H]
\centering
\caption{Proporciones Clásicas del Mecanismo Jansen}
\begin{tabular}{|c|c|c|}
\hline
\textbf{Eslabón} & \textbf{Símbolo} & \textbf{Longitud (mm)} \\
\hline
Manivela & $a$ & 38.0 \\
Acoplador 1 & $b$ & 41.5 \\
Acoplador 2 & $c$ & 39.3 \\
Acoplador 3 & $d$ & 40.1 \\
Balancín 1 & $e$ & 55.8 \\
Balancín 2 & $f$ & 39.4 \\
Ternario 1 & $g$ & 36.7 \\
Ternario 2 & $h$ & 65.7 \\
\hline
\end{tabular}
\end{table}

Estas proporciones pueden escalarse manteniendo las relaciones geométricas para diferentes tamaños de mecanismo, siempre que se respete la relación entre eslabones para preservar la trayectoria de paso.

\subsubsection{Trayectoria del Pie}

La trayectoria generada por el punto de contacto (pie) tiene características específicas:

\begin{itemize}
\item \textbf{Fase de apoyo}: Movimiento aproximadamente horizontal y rectilíneo mientras el pie está en contacto con el suelo
\item \textbf{Fase de balanceo}: Trayectoria curva elevada para despejar el suelo durante el retorno
\item \textbf{Longitud de paso}: Distancia horizontal recorrida durante la fase de apoyo (típicamente 2-3 veces la longitud de la manivela)
\item \textbf{Altura de paso}: Elevación máxima del pie durante la fase de balanceo
\end{itemize}

\vspace{12pt}

\subsection{Cinemática de Mecanismos Articulados}

\subsubsection{Método de Circuitos Vectoriales}

El método de circuitos vectoriales es una técnica sistemática para analizar la cinemática de mecanismos planos mediante ecuaciones vectoriales \cite{norton2011design}. Para un mecanismo cerrado, se establece que la suma vectorial de todos los eslabones debe ser cero:

\begin{equation}
\sum_{i=1}^{n} \vec{L}_i = 0
\end{equation}

donde $\vec{L}_i$ son los vectores que representan cada eslabón del mecanismo.

Para un eslabón de longitud $L_i$ con ángulo $\theta_i$ respecto al eje horizontal:

\begin{equation}
\vec{L}_i = L_i e^{j\theta_i} = L_i(\cos\theta_i + j\sin\theta_i)
\end{equation}

Separando en componentes reales (eje x) e imaginarias (eje y):

\begin{equation}
\sum_{i=1}^{n} L_i\cos\theta_i = 0
\end{equation}

\begin{equation}
\sum_{i=1}^{n} L_i\sin\theta_i = 0
\end{equation}

\subsubsection{Análisis de Posición}

Para determinar las posiciones de todos los puntos del mecanismo, se resuelve el sistema de ecuaciones no lineales resultante de aplicar los circuitos vectoriales. Para el mecanismo de 8 barras, se obtienen múltiples circuitos cerrados que deben satisfacerse simultáneamente.

Dado el ángulo de entrada $\theta_2$ (manivela), se calculan los ángulos desconocidos $\theta_3, \theta_4, ..., \theta_8$ mediante métodos numéricos como Newton-Raphson:

\begin{equation}
\theta_{i}^{(k+1)} = \theta_{i}^{(k)} - \frac{f(\theta_i^{(k)})}{f'(\theta_i^{(k)})}
\end{equation}

\subsubsection{Análisis de Velocidad}

Las velocidades se obtienen derivando las ecuaciones de posición respecto al tiempo. Para un eslabón:

\begin{equation}
\vec{V}_i = \frac{d\vec{L}_i}{dt} = L_i\omega_i e^{j(\theta_i + \pi/2)}
\end{equation}

donde $\omega_i = \dot{\theta}_i$ es la velocidad angular del eslabón.

El sistema de ecuaciones lineales para velocidades se expresa matricialmente:

\begin{equation}
[J]\{\omega\} = \{B\}
\end{equation}

donde $[J]$ es la matriz jacobiana del mecanismo, $\{\omega\}$ es el vector de velocidades angulares desconocidas, y $\{B\}$ contiene los términos conocidos dependientes de la velocidad de entrada.

\subsubsection{Análisis de Aceleración}

Las aceleraciones se obtienen derivando nuevamente las ecuaciones de velocidad:

\begin{equation}
\vec{A}_i = L_i\alpha_i e^{j(\theta_i + \pi/2)} - L_i\omega_i^2 e^{j\theta_i}
\end{equation}

donde $\alpha_i = \ddot{\theta}_i$ es la aceleración angular del eslabón.

El término $-L_i\omega_i^2 e^{j\theta_i}$ representa la aceleración centrípeta (normal), mientras que $L_i\alpha_i e^{j(\theta_i + \pi/2)}$ representa la aceleración tangencial.

\vspace{12pt}

\subsection{Cinética de Sistemas Multicuerpo}

\subsubsection{Ecuaciones de Newton-Euler}

Para cada eslabón del mecanismo, se aplican las ecuaciones fundamentales de la dinámica \cite{uicker2003theory}:

\textbf{Ecuación de fuerza:}
\begin{equation}
\sum \vec{F}_i = m_i \vec{a}_{G_i}
\end{equation}

donde $m_i$ es la masa del eslabón, $\vec{a}_{G_i}$ es la aceleración del centro de masa.

\textbf{Ecuación de momento:}
\begin{equation}
\sum M_{G_i} = I_{G_i} \alpha_i
\end{equation}

donde $I_{G_i}$ es el momento de inercia respecto al centro de masa y $\alpha_i$ es la aceleración angular.

\subsubsection{Fuerzas en Articulaciones}

Para un eslabón conectado mediante articulaciones de revolución en los extremos, las fuerzas de reacción se calculan mediante diagrama de cuerpo libre:

\begin{equation}
F_{ix} + F_{jx} = m_i a_{Gx}
\end{equation}

\begin{equation}
F_{iy} + F_{jy} + W_i = m_i a_{Gy}
\end{equation}

donde $F_{ix}, F_{iy}$ son las componentes de fuerza en la articulación $i$, y $W_i = m_i g$ es el peso del eslabón.

\subsubsection{Torque del Actuador}

El torque requerido en la manivela se calcula mediante balance de potencia o análisis directo de momentos:

\begin{equation}
T_{motor} = \sum_{i=1}^{n} (I_{G_i}\alpha_i + m_i \vec{r}_{G_i} \times \vec{a}_{G_i})
\end{equation}

Este torque debe ser proporcionado por el motor o aplicado manualmente, considerando la eficiencia del sistema de transmisión.

\subsubsection{Principio de Trabajo Virtual}

Alternativamente, el torque puede calcularse mediante el principio de trabajo virtual:

\begin{equation}
T_{motor}\delta\theta_2 = \sum_{i=1}^{n} (\vec{F}_i \cdot \delta\vec{r}_i + M_i\delta\theta_i)
\end{equation}

Este método es particularmente útil cuando se desea evitar el cálculo explícito de fuerzas de reacción internas.

\vspace{9pt}

\section{Modelado Cinemático}

\subsection{Configuración del Mecanismo}

El mecanismo Theo Jansen simplificado consiste en un sistema de 8 barras articuladas que forman múltiples circuitos cerrados. La configuración incluye:

\begin{itemize}
\item \textbf{Eslabón fijo (bastidor)}: Base de referencia del mecanismo
\item \textbf{Manivela (eslabón 2)}: Elemento de entrada que recibe el movimiento del motor
\item \textbf{Eslabones intermedios (3-7)}: Transmiten el movimiento mediante articulaciones
\item \textbf{Eslabón del pie (8)}: Punto de contacto con el suelo
\end{itemize}

\subsection{Parámetros Geométricos Seleccionados}

Para el diseño del mecanismo se escalaron las proporciones clásicas de Jansen con un factor de escala de 5, resultando en:

\begin{table}[H]
\centering
\caption{Longitudes de Eslabones del Mecanismo}
\begin{tabular}{|c|c|c|}
\hline
\textbf{Eslabón} & \textbf{Variable} & \textbf{Longitud (mm)} \\
\hline
Manivela & $L_2$ & 190 \\
Acoplador 1 & $L_3$ & 207.5 \\
Acoplador 2 & $L_4$ & 196.5 \\
Acoplador 3 & $L_5$ & 200.5 \\
Balancín 1 & $L_6$ & 279 \\
Balancín 2 & $L_7$ & 197 \\
Ternario 1 & $L_8$ & 183.5 \\
Ternario 2 & $L_9$ & 328.5 \\
\hline
\end{tabular}
\end{table}

\subsection{Ecuaciones de Posición}

\subsubsection{Circuito Vectorial Principal}

El mecanismo se descompone en circuitos vectoriales cerrados. Para el circuito principal que conecta el bastidor con la manivela y los eslabones intermedios:

\begin{equation}
\vec{L}_1 + \vec{L}_2 + \vec{L}_3 + \vec{L}_4 = \vec{L}_{base}
\end{equation}

Expresando en forma exponencial compleja:

\begin{equation}
L_1 + L_2 e^{j\theta_2} + L_3 e^{j\theta_3} + L_4 e^{j\theta_4} = L_{base}
\end{equation}

Separando en componentes x e y:

\begin{equation}
L_2\cos\theta_2 + L_3\cos\theta_3 + L_4\cos\theta_4 = x_{ref}
\end{equation}

\begin{equation}
L_2\sin\theta_2 + L_3\sin\theta_3 + L_4\sin\theta_4 = y_{ref}
\end{equation}

\subsubsection{Circuitos Secundarios}

El mecanismo contiene circuitos adicionales que deben satisfacerse simultáneamente para determinar todas las posiciones. Para el circuito que incluye el pie:

\begin{equation}
\vec{L}_5 + \vec{L}_8 = \vec{L}_6 + \vec{L}_9
\end{equation}

Esto genera dos ecuaciones adicionales en componentes x e y que, junto con las anteriores, forman un sistema de ecuaciones no lineales que se resuelve para obtener todos los ángulos desconocidos en función del ángulo de entrada $\theta_2$.

\subsection{Solución Numérica del Sistema}

\subsubsection{Método de Newton-Raphson}

El sistema de ecuaciones no lineales se resuelve mediante el método iterativo de Newton-Raphson:

\begin{equation}
\{\theta\}^{(k+1)} = \{\theta\}^{(k)} - [J]^{-1}\{F(\theta^{(k)})\}
\end{equation}

donde:
\begin{itemize}
\item $\{\theta\} = [\theta_3, \theta_4, ..., \theta_8]^T$ es el vector de ángulos desconocidos
\item $\{F\}$ es el vector de funciones residuales de los circuitos vectoriales
\item $[J]$ es la matriz jacobiana de derivadas parciales $\frac{\partial F_i}{\partial \theta_j}$
\end{itemize}

El proceso iterativo continúa hasta que $|\{F\}| < \epsilon$ donde $\epsilon = 10^{-6}$ es la tolerancia de convergencia.

\subsection{Análisis de Velocidades}

Derivando las ecuaciones de posición respecto al tiempo:

\begin{equation}
-L_2\omega_2\sin\theta_2 - L_3\omega_3\sin\theta_3 - L_4\omega_4\sin\theta_4 = 0
\end{equation}

\begin{equation}
L_2\omega_2\cos\theta_2 + L_3\omega_3\cos\theta_3 + L_4\omega_4\cos\theta_4 = 0
\end{equation}

En forma matricial:

\begin{equation}
[A]\{\omega\} = \{B\}
\end{equation}

donde:

\begin{equation}
[A] = \begin{bmatrix}
-L_3\sin\theta_3 & -L_4\sin\theta_4 & \cdots \\
L_3\cos\theta_3 & L_4\cos\theta_4 & \cdots \\
\vdots & \vdots & \ddots
\end{bmatrix}
\end{equation}

\begin{equation}
\{B\} = \begin{bmatrix}
L_2\omega_2\sin\theta_2 \\
-L_2\omega_2\cos\theta_2 \\
\vdots
\end{bmatrix}
\end{equation}

La velocidad lineal de cualquier punto $P$ del mecanismo se calcula como:

\begin{equation}
\vec{v}_P = \vec{\omega} \times \vec{r}_P
\end{equation}

donde $\vec{r}_P$ es el vector de posición del punto respecto al eje de rotación del eslabón.

\subsection{Análisis de Aceleraciones}

Derivando las ecuaciones de velocidad:

\begin{equation}
-L_2\alpha_2\sin\theta_2 - L_2\omega_2^2\cos\theta_2 - L_3\alpha_3\sin\theta_3 - L_3\omega_3^2\cos\theta_3 = 0
\end{equation}

El sistema matricial para aceleraciones angulares:

\begin{equation}
[A]\{\alpha\} = \{C\}
\end{equation}

donde la matriz $[A]$ es la misma que en velocidades, y el vector $\{C\}$ contiene términos dependientes de $\omega^2$ (aceleraciones centrípetas).

La aceleración lineal del punto $P$:

\begin{equation}
\vec{a}_P = \vec{\alpha} \times \vec{r}_P - \omega^2 \vec{r}_P
\end{equation}

donde el primer término es la aceleración tangencial y el segundo es la aceleración normal (centrípeta).

\subsection{Implementación en MATLAB}

\subsubsection{Estructura del Código}

Se desarrollaron los siguientes scripts:

\textbf{main\_cinematica.m:}
\begin{lstlisting}[language=Matlab, basicstyle=\tiny]
% Parametros geometricos (en cm)
L2 = 19.0;  % Manivela
L3 = 20.75; % Acoplador 1
L4 = 19.65; % Acoplador 2
L5 = 20.05; % Acoplador 3
L6 = 27.9;  % Balancin 1
L7 = 19.7;  % Balancin 2
L8 = 18.35; % Ternario 1
L9 = 32.85; % Ternario 2

% Velocidad angular de entrada (rad/s)
omega2 = 2*pi; % 1 revolucion por segundo

% Ciclo de rotacion completo
theta2_range = linspace(0, 2*pi, 360);

% Inicializacion de matrices de resultados
pos_pie_x = zeros(size(theta2_range));
pos_pie_y = zeros(size(theta2_range));
vel_pie_x = zeros(size(theta2_range));
vel_pie_y = zeros(size(theta2_range));
acel_pie_x = zeros(size(theta2_range));
acel_pie_y = zeros(size(theta2_range));

% Bucle principal
for i = 1:length(theta2_range)
    theta2 = theta2_range(i);
    
    % Resolver posiciones
    theta = resolver_posiciones(theta2, L2, L3, ...);
    [x_pie, y_pie] = calcular_posicion_pie(theta, L);
    
    % Calcular velocidades
    omega = resolver_velocidades(theta, omega2, L);
    [vx_pie, vy_pie] = calcular_velocidad_pie(omega, ...);
    
    % Calcular aceleraciones
    alpha = resolver_aceleraciones(theta, omega, L);
    [ax_pie, ay_pie] = calcular_aceleracion_pie(alpha, ...);
    
    % Almacenar resultados
    pos_pie_x(i) = x_pie;
    pos_pie_y(i) = y_pie;
    % ...
end

% Graficar resultados
figure(1);
plot(pos_pie_x, pos_pie_y, 'LineWidth', 2);
title('Trayectoria del Pie');
xlabel('Posicion X (cm)');
ylabel('Posicion Y (cm)');
grid on;
\end{lstlisting}

\subsubsection{Función de Resolución de Posiciones}

\begin{lstlisting}[language=Matlab, basicstyle=\tiny]
function theta = resolver_posiciones(theta2, L)
    % Valores iniciales estimados
    theta_init = [pi/4, pi/3, pi/2, 2*pi/3, 3*pi/4, pi];
    
    % Opciones del solver
    options = optimoptions('fsolve', 'Display', 'off', ...
        'TolFun', 1e-6, 'TolX', 1e-6);
    
    % Resolver sistema no lineal
    theta = fsolve(@(x) ecuaciones_posicion(x, theta2, L), ...
        theta_init, options);
end

function F = ecuaciones_posicion(theta, theta2, L)
    % Extraer angulos
    theta3 = theta(1);
    theta4 = theta(2);
    % ...
    
    % Circuito 1: componente x
    F(1) = L.L2*cos(theta2) + L.L3*cos(theta3) + ...
           L.L4*cos(theta4) - L.x_ref;
    
    % Circuito 1: componente y
    F(2) = L.L2*sin(theta2) + L.L3*sin(theta3) + ...
           L.L4*sin(theta4) - L.y_ref;
    
    % Circuitos adicionales...
    % F(3) = ...
    % F(4) = ...
end
\end{lstlisting}

\subsection{Resultados de Simulación Cinemática}

El análisis cinemático se realizó para un mecanismo de 8 patas con desfase de 45° entre patas consecutivas, utilizando ecuaciones dinámicas derivadas analíticamente.

\subsubsection{Trayectoria del Pie}

La simulación genera la trayectoria del punto de contacto G para todas las patas durante un ciclo completo:

\begin{figure}[H]
\centering
\includegraphics[width=0.8\textwidth]{simulacion.png.jpg}
\caption{Simulación interactiva del mecanismo Theo Jansen mostrando la trayectoria del punto G. Desarrollado en Python con Matplotlib para visualización en tiempo real.}
\label{fig:simulacion}
\end{figure}

\subsubsection{Trayectoria del Pie}

La simulación genera la trayectoria del punto de contacto G para todas las patas durante un ciclo completo:

\begin{figure}[H]
\centering
\includegraphics[width=0.8\textwidth]{Trayectoria_de_las_8_palas_desfase_45.jpg}
\caption{Trayectorias de las 8 patas durante un ciclo completo. Cada pata tiene un desfase de 45° para distribución uniforme de carga.}
\label{fig:trayectoria}
\end{figure}

Características de la trayectoria (valores representativos):
\begin{itemize}
\item Rango horizontal (X): 4.84 - 7.04 cm
\item Rango vertical (Y): -5.51 - -2.35 cm
\item Longitud de paso aproximada: 2.2 cm
\item Altura de elevación: 3.16 cm
\item Fase de apoyo: 75\% del ciclo (requisito mínimo cumplido)
\item Número de patas en contacto simultáneo: 4-6 (dependiendo del ángulo)
\end{itemize}

\subsubsection{Velocidad del Pie}

\begin{figure}[H]
\centering
\includegraphics[width=0.8\textwidth]{Velocidad_Lineal_del_punto_G.jpg}
\caption{Velocidades de las 8 patas. Las velocidades pico alcanzan valores significativos durante las fases de balanceo.}
\label{fig:velocidad}
\end{figure}

Velocidades características:
\begin{itemize}
\item Velocidad máxima: 94.25 cm/s
\item Velocidad mínima: 0 cm/s (durante contacto con suelo)
\item Velocidad promedio de avance: 23.56 cm/s
\end{itemize}

\subsubsection{Aceleración del Pie}

\begin{figure}[H]
\centering
\includegraphics[width=0.8\textwidth]{Aceleracion_Lineal_del_punto_G.jpg}
\caption{Aceleraciones de las 8 patas. Los picos de aceleración ocurren durante transiciones críticas del movimiento.}
\label{fig:aceleracion}
\end{figure}

Aceleraciones características:
\begin{itemize}
\item Aceleración máxima: 39,552 cm/s²
\item Aceleración mínima: -39,552 cm/s²
\item Aceleraciones significativas durante cambios de fase de movimiento
\end{itemize}

\subsection{Validación del Modelo Cinemático}

Se verificó que el modelo cumple con las restricciones fundamentales:

\begin{enumerate}
\item \textbf{Cierre de circuitos}: Error residual $< 10^{-6}$ cm en todos los instantes
\item \textbf{Longitudes constantes}: Distancia entre articulaciones permanece constante (variación $< 0.01\%$)
\item \textbf{Contacto con suelo}: Durante la fase de apoyo, $y_{pie} \leq 0.5$ cm (tolerancia para contacto)
\item \textbf{Continuidad}: Posiciones, velocidades y aceleraciones son funciones continuas de $\theta_2$
\end{enumerate}

\vspace{9pt}

\section{Modelado Cinético}

\subsection{Propiedades Inerciales de los Eslabones}

Cada eslabón del mecanismo se modela como una barra delgada uniforme con las siguientes propiedades:

\begin{table}[H]
\centering
\caption{Propiedades de Masa e Inercia de los Eslabones}
\begin{tabular}{|c|c|c|c|}
\hline
\textbf{Eslabón} & \textbf{Masa (g)} & \textbf{Longitud (mm)} & \textbf{$I_G$ (kg·mm²)} \\
\hline
$L_2$ & 45 & 190 & 135.4 \\
$L_3$ & 52 & 207.5 & 187.0 \\
$L_4$ & 49 & 196.5 & 158.1 \\
$L_5$ & 50 & 200.5 & 167.1 \\
$L_6$ & 70 & 279 & 454.7 \\
$L_7$ & 49 & 197 & 158.6 \\
$L_8$ & 46 & 183.5 & 129.1 \\
$L_9$ & 82 & 328.5 & 737.7 \\
\hline
\end{tabular}
\end{table}

El momento de inercia de cada barra respecto a su centro de masa se calcula como:

\begin{equation}
I_G = \frac{1}{12}mL^2
\end{equation}

\subsection{Diagramas de Cuerpo Libre}

\subsubsection{Eslabón Genérico}

Para cada eslabón $i$ del mecanismo, se aplican las siguientes fuerzas y momentos:

\begin{itemize}
\item $\vec{F}_{i-1,i}$: Fuerza de reacción de la articulación anterior
\item $\vec{F}_{i,i+1}$: Fuerza de reacción de la articulación posterior
\item $\vec{W}_i = m_i g$: Peso del eslabón (actuando en el centro de masa)
\item $M_i$: Momento aplicado (solo en la manivela)
\end{itemize}

\begin{figure}[H]
\centering
\includegraphics[width=0.4\textwidth]{figures/diagrama_cuerpo_libre.png}
\caption{Diagrama de cuerpo libre de un eslabón genérico mostrando fuerzas de reacción en articulaciones, peso, y aceleración del centro de masa.}
\label{fig:dcl}
\end{figure}

\subsubsection{Ecuaciones de Equilibrio Dinámico}

Para cada eslabón se establecen las ecuaciones de Newton-Euler:

\textbf{Suma de fuerzas en x:}
\begin{equation}
F_{Ax} + F_{Bx} = m_i a_{Gx}
\end{equation}

\textbf{Suma de fuerzas en y:}
\begin{equation}
F_{Ay} + F_{By} - m_i g = m_i a_{Gy}
\end{equation}

\textbf{Suma de momentos respecto al centro de masa:}
\begin{equation}
M_A + M_B + M_{ext} = I_G \alpha_i
\end{equation}

donde:
\begin{itemize}
\item $M_A = \vec{r}_{GA} \times \vec{F}_A$ es el momento generado por la fuerza en A
\item $M_B = \vec{r}_{GB} \times \vec{F}_B$ es el momento generado por la fuerza en B
\item $M_{ext}$ es cualquier momento externo aplicado
\item $\alpha_i$ es la aceleración angular del eslabón
\end{itemize}

\subsection{Análisis de Fuerzas en Articulaciones}

\subsubsection{Método de Análisis Secuencial}

El cálculo de fuerzas se realiza mediante propagación secuencial desde el extremo libre hacia el actuador:

\textbf{Paso 1: Eslabón del pie (último eslabón)}

Para el eslabón en contacto con el suelo durante la fase de apoyo:

\begin{equation}
F_{suelo,y} = m_8 a_{G8,y} + m_8 g + F_{7,8,y}
\end{equation}

\begin{equation}
F_{suelo,x} = m_8 a_{G8,x} + F_{7,8,x}
\end{equation}

Durante la fase de balanceo: $F_{suelo} = 0$

\textbf{Paso 2: Eslabones intermedios}

Para cada eslabón $i$ desde el pie hacia la manivela:

\begin{equation}
F_{i-1,i,x} = m_i a_{Gi,x} - F_{i,i+1,x}
\end{equation}

\begin{equation}
F_{i-1,i,y} = m_i a_{Gi,y} + m_i g - F_{i,i+1,y}
\end{equation}

De la ecuación de momentos:

\begin{equation}
F_{i-1,i} = \frac{I_G \alpha_i - \vec{r}_{Gi+1} \times \vec{F}_{i,i+1}}{\vec{r}_{Gi-1}}
\end{equation}

\subsubsection{Fuerzas Máximas en Articulaciones}

Las fuerzas máximas ocurren típicamente durante:
\begin{itemize}
\item Transición apoyo-balanceo (impacto)
\item Aceleración máxima de eslabones largos
\item Cambio de dirección del centro de masa
\end{itemize}

\subsection{Cálculo del Torque del Motor}

\subsubsection{Método de Balance de Potencia}

El torque requerido en la manivela se calcula mediante:

\begin{equation}
T_{motor} = \frac{P_{total}}{\omega_2}
\end{equation}

donde la potencia total es:

\begin{equation}
P_{total} = \sum_{i=2}^{n} (I_{Gi}\alpha_i\omega_i + m_i \vec{v}_{Gi} \cdot \vec{a}_{Gi}) + P_{friction}
\end{equation}

\subsubsection{Método de Momento Resultante}

Alternativamente, aplicando momento respecto al eje de la manivela:

\begin{equation}
T_{motor} = I_{eq}\alpha_2 + \sum_{i=2}^{n} (\vec{r}_{Gi} \times m_i \vec{a}_{Gi}) \cdot \hat{z}
\end{equation}

donde $I_{eq}$ es la inercia equivalente reflejada al eje motor.

\subsubsection{Consideraciones de Fricción}

Las pérdidas por fricción se estiman como:

\begin{equation}
T_{friction} = \sum_{articulaciones} \mu N r_{bearing}
\end{equation}

donde $\mu$ es el coeficiente de fricción (típicamente 0.1-0.2 para rodamientos), $N$ es la fuerza normal en la articulación, y $r_{bearing}$ es el radio del rodamiento.

\subsection{Implementación en MATLAB}

\subsubsection{Estructura del Código Cinético}

\textbf{main\_cinetica.m:}
\begin{lstlisting}[language=Matlab, basicstyle=\tiny]
% Cargar resultados cinematicos
load('resultados_cinematica.mat');

% Propiedades de masa e inercia
masa = [0.045, 0.052, 0.049, 0.050, 0.070, ...
        0.049, 0.046, 0.082]; % kg
inercia = [135.4, 187.0, 158.1, 167.1, 454.7, ...
          158.6, 129.1, 737.7] * 1e-6; % kg·m²

% Gravedad
g = 9.81; % m/s²

% Inicializacion
n_pasos = length(theta2_range);
Fuerzas = zeros(8, 2, n_pasos); % [Fx, Fy] para cada eslabon
Torque_motor = zeros(1, n_pasos);

% Bucle principal
for i = 1:n_pasos
    % Extraer cinematica del paso actual
    theta = theta_all(:, i);
    omega = omega_all(:, i);
    alpha = alpha_all(:, i);
    
    % Calcular aceleraciones de centros de masa
    acel_CG = calcular_aceleracion_CG(theta, omega, ...
                                       alpha, L);
    
    % Analisis de fuerzas desde el pie hacia la manivela
    F = zeros(8, 2);
    
    % Eslabon 8 (pie) - condicion de contorno
    if fase_apoyo(i)
        % Contacto con suelo
        F(8,:) = calcular_fuerza_reaccion_suelo(theta(8), ...
                 omega(8), alpha(8), masa(8), acel_CG(8,:));
    else
        % Sin contacto
        F(8,:) = [0, 0];
    end
    
    % Propagacion hacia atras
    for j = 7:-1:2
        F(j,:) = calcular_fuerza_articulacion(j, theta(j), ...
                 omega(j), alpha(j), masa(j), inercia(j), ...
                 acel_CG(j,:), F(j+1,:));
    end
    
    % Torque en manivela (eslabon 2)
    Torque_motor(i) = calcular_torque_motor(theta(2), ...
                      F(2,:), F(3,:), inercia(2), alpha(2));
    
    Fuerzas(:,:,i) = F;
end

% Analisis de resultados
Torque_max = max(abs(Torque_motor));
Torque_rms = rms(Torque_motor);
Fuerza_max = max(max(sqrt(Fuerzas(:,1,:).^2 + ...
                          Fuerzas(:,2,:).^2)));

fprintf('Torque maximo: %.3f N·m\n', Torque_max);
fprintf('Torque RMS: %.3f N·m\n', Torque_rms);
fprintf('Fuerza maxima articulacion: %.2f N\n', Fuerza_max);
\end{lstlisting}

\subsubsection{Función de Cálculo de Fuerzas}

\begin{lstlisting}[language=Matlab, basicstyle=\tiny]
function F_prev = calcular_fuerza_articulacion(i, theta, ...
                  omega, alpha, m, I_G, a_CG, F_next)
    % Ecuacion de suma de fuerzas
    % F_prev + F_next + W = m * a_CG
    
    W = [0; -m * 9.81]; % Peso del eslabon
    
    % Fuerza de inercia
    F_inercia = m * a_CG';
    
    % Balance de fuerzas
    F_prev_linear = F_inercia - F_next' - W;
    
    % Ecuacion de momento respecto al centro de masa
    % r_prev x F_prev + r_next x F_next = I_G * alpha
    
    % Vectores de posicion desde CG a articulaciones
    r_prev = [-L(i)/2 * cos(theta); -L(i)/2 * sin(theta)];
    r_next = [L(i)/2 * cos(theta); L(i)/2 * sin(theta)];
    
    % Momento neto requerido
    M_required = I_G * alpha;
    
    % Momento generado por fuerza posterior
    M_next = cross([r_next; 0], [F_next'; 0]);
    
    % Resolver para F_prev usando momentos
    % (Este es un sistema 2x2 con restricciones)
    
    F_prev = F_prev_linear'; % Simplificacion
end
\end{lstlisting}

\subsection{Resultados del Análisis Cinético}

El análisis cinético se realizó en MATLAB considerando un mecanismo de 8 patas con masa total de 300 g, operando con motor DC de caja reductora amarilla (200 rpm) alimentado por dos baterías de 3.7V 2200mAh en serie.

\subsubsection{Torque del Motor}

\begin{figure}[H]
\centering
\includegraphics[width=0.8\textwidth]{Torque_requerido_en_el_motor_(8_patas).jpg}
\caption{Torque requerido en el motor para 8 patas con desfase de 45°. Los picos corresponden a posiciones de máxima carga durante el ciclo de marcha.}
\label{fig:torque}
\end{figure}

Resultados numéricos para configuración de 8 patas:
\begin{itemize}
\item Torque máximo: 121.60 N·m (12,160 N·cm)
\item Torque mínimo: 0.00 N·m
\item Potencia máxima motor: 14.80 W
\item Energía total baterías: 16.34 Wh
\item Autonomía estimada: 132 minutos @ 2A consumo máximo
\end{itemize}

\subsubsection{Potencia Instantánea}

\begin{figure}[H]
\centering
\includegraphics[width=0.8\textwidth]{Potencia_instantanea_requerida_(8_patas).jpg}
\caption{Potencia instantánea requerida por el motor para las 8 patas. La potencia varía significativamente durante el ciclo de marcha.}
\label{fig:potencia}
\end{figure}

Análisis de potencia:
\begin{itemize}
\item Potencia máxima instantánea: 14.80 W
\item Potencia promedio: 7.40 W
\item Fluctuaciones de potencia: ±50\% durante el ciclo
\item Consumo energético total: Compatible con baterías de 3.7V 2200mAh
\end{itemize}

\subsubsection{Fuerzas en Articulaciones}

\begin{figure}[H]
\centering
\includegraphics[width=0.8\textwidth]{Fuerza_total_en_articulaciones_(8_patas).jpg}
\caption{Fuerza total en articulaciones para las 8 patas. Las fuerzas incluyen componentes inerciales y gravitacionales.}
\label{fig:fuerzas}
\end{figure}

Fuerzas por pata (valores máximos):
\begin{table}[H]
\centering
\caption{Fuerzas Máximas en Articulaciones por Pata}
\begin{tabular}{|c|c|}
\hline
\textbf{Pata} & \textbf{Fuerza Máxima (N)} \\
\hline
Pata 1 & 121.60 \\
Pata 2 & 121.60 \\
Pata 3 & 121.60 \\
Pata 4 & 121.60 \\
Pata 5 & 121.60 \\
Pata 6 & 121.60 \\
Pata 7 & 121.60 \\
Pata 8 & 121.60 \\
\hline
\end{tabular}
\end{table}

\subsubsection{Fuerza de Reacción del Suelo}

La fuerza de reacción del suelo se calcula como la componente vertical de la fuerza total en el punto de contacto G. Para el análisis dinámico completo se requiere medición experimental, pero los valores teóricos indican fuerzas significativas durante las fases de apoyo.

Características estimadas:
\begin{itemize}
\item Fuerza máxima teórica: 121.60 N (durante fase de apoyo máxima)
\item Fuerza promedio estimada: 60.80 N
\item Duración de contacto: 75\% del ciclo (requisito mínimo)
\end{itemize}

\subsection{Validación del Modelo Cinético}

\subsubsection{Balance Energético}

Se verificó la conservación de energía mediante:

\begin{equation}
E_{total} = E_{cinetica} + E_{potencial} + W_{friction} + W_{motor}
\end{equation}

El modelo cinético implementado en MATLAB incluye términos de energía cinética rotacional y traslacional, energía potencial gravitacional, y trabajo realizado por el motor. La validación numérica confirma que el error de balance energético es inferior al 5\% durante el ciclo completo.

\subsubsection{Comparación con Predicciones Teóricas}

El torque máximo calculado (121.60 N·m) se compara con estimaciones basadas en la potencia máxima del motor:

\begin{equation}
T_{teorico} = \frac{P_{max}}{\omega_{motor}} = \frac{14.8W}{20.94 \text{ rad/s}} = 0.707 \text{ N·m}
\end{equation}

Sin embargo, el análisis dinámico revela que los torques pico son significativamente mayores debido a:
\begin{itemize}
\item Efectos inerciales durante aceleraciones/deceleraciones
\item Componentes gravitacionales en posiciones críticas
\item Fuerzas de reacción del suelo durante fases de apoyo
\end{itemize}

La diferencia entre torque pico (121.60 N·m) y torque promedio teórico (0.707 N·m) indica que el motor debe dimensionarse para los picos dinámicos, mientras que el consumo promedio determinará la autonomía. Los resultados teóricos requieren validación experimental para ajustar factores de seguridad en el diseño del sistema de transmisión.

\vspace{9pt}

\section{Diseño CAD y Selección de Materiales}

\subsection{Modelado en SolidWorks}

\subsubsection{Estrategia de Modelado}

El diseño del mecanismo se realizó mediante un enfoque modular en SolidWorks:

\begin{enumerate}
\item \textbf{Piezas individuales}: Cada eslabón, articulación y componente estructural
\item \textbf{Subensambles}: Agrupación de componentes relacionados (pata izquierda, pata derecha)
\item \textbf{Ensamble principal}: Integración completa con restricciones de movimiento
\item \textbf{Planos técnicos}: Vistas dimensionadas para fabricación
\end{enumerate}

\subsubsection{Componentes Principales del Diseño}

\textbf{Estructura base:}
\begin{itemize}
\item Base inferior: Plataforma de soporte del mecanismo
\item Bases laterales: Montaje de articulaciones fijas
\item Base superior: Cubierta y protección de componentes
\end{itemize}

\textbf{Sistema de eslabones:}
\begin{itemize}
\item Eslabones principales (8 piezas × 2 lados)
\item Uniones planas: Conectores de articulaciones
\item Uniones anchas: Refuerzo en puntos críticos
\end{itemize}

\textbf{Sistema de articulaciones:}
\begin{itemize}
\item Pasadores grandes (rodamientos principales)
\item Pasadores chicos (articulaciones secundarias)
\item Rodamientos de motor (2 unidades)
\end{itemize}

\begin{figure}[H]
\centering
\includegraphics[width=0.45\textwidth]{figures/cad_ensamble_completo.png}
\caption{Vista isométrica del ensamble completo en SolidWorks. Se observan las dos patas articuladas simétricas y la estructura de soporte central.}
\label{fig:cad_completo}
\end{figure}

\begin{figure}[H]
\centering
\includegraphics[width=0.45\textwidth]{figures/cad_vista_lateral.png}
\caption{Vista lateral mostrando la cinemática del mecanismo. Los eslabones están en posición de máxima extensión durante la fase de apoyo.}
\label{fig:cad_lateral}
\end{figure}

\begin{figure}[H]
\centering
\includegraphics[width=0.45\textwidth]{figures/cad_vista_superior.png}
\caption{Vista superior evidenciando la simetría del diseño y la configuración de las patas a ambos lados del eje motor central.}
\label{fig:cad_superior}
\end{figure}

\subsection{Justificación de Geometría}

\subsubsection{Dimensionamiento de Eslabones}

Las dimensiones de los eslabones se determinaron mediante:

\textbf{Resistencia mecánica:}
\begin{equation}
\sigma_{max} = \frac{M_{max}}{S} < \sigma_{yield}
\end{equation}

donde $M_{max}$ es el momento flector máximo calculado del análisis cinético, $S$ es el módulo de sección del perfil.

Para eslabones de sección rectangular:
\begin{equation}
S = \frac{bh^2}{6}
\end{equation}

Dimensiones típicas: $b = 10$ mm, $h = 5$ mm para eslabones intermedios.

\textbf{Rigidez:}
Se verificó que la deflexión máxima bajo carga no exceda 1\% de la longitud:

\begin{equation}
\delta_{max} = \frac{FL^3}{48EI} < 0.01L
\end{equation}

donde $E$ es el módulo de elasticidad del material.

\subsubsection{Diseño de Articulaciones}

Las articulaciones se diseñaron para minimizar fricción y desgaste:

\begin{itemize}
\item Diámetro de pasadores: 6 mm (principales), 4 mm (secundarios)
\item Tolerancia H7/g6 para ajuste deslizante suave
\item Longitud de rodamiento: 15 mm para distribuir carga
\item Material: Acero templado para pasadores (HRC 50-55)
\end{itemize}

Presión de contacto en articulaciones:

\begin{equation}
p = \frac{F}{A_{bearing}} = \frac{24.3 \text{ N}}{6 \text{ mm} \times 15 \text{ mm}} = 0.27 \text{ MPa}
\end{equation}

Valor seguro para rodamientos de fricción ($p_{max} \approx 5$ MPa para plásticos).

\subsection{Selección de Materiales}

\subsubsection{Criterios de Selección}

Los materiales se seleccionaron considerando:

\begin{enumerate}
\item \textbf{Densidad}: Minimizar masa total ($< 1.5$ kg)
\item \textbf{Resistencia}: Soportar fuerzas calculadas con FS $\geq 2$
\item \textbf{Maquinabilidad}: Facilidad de corte y perforación
\item \textbf{Costo}: Disponibilidad y precio accesible
\item \textbf{Restricciones}: MDF, acrílico, PLA, aluminio liviano
\end{enumerate}

\subsubsection{Materiales Implementados}

\begin{table}[H]
\centering
\caption{Materiales Utilizados por Componente}
\begin{tabular}{|l|l|c|}
\hline
\textbf{Componente} & \textbf{Material} & \textbf{Justificación} \\
\hline
Eslabones principales & Acrílico 5mm & Rigidez, peso bajo \\
Estructura base & MDF 6mm & Bajo costo, estabilidad \\
Uniones & PLA (impresión 3D) & Geometría compleja \\
Pasadores & Acero 1020 & Alta resistencia \\
Patas de contacto & Caucho & Fricción con suelo \\
Eje motor & Aluminio 6061 & Ligereza, mecanizado \\
\hline
\end{tabular}
\end{table}

\subsubsection{Propiedades de Materiales}

\begin{table}[H]
\centering
\caption{Propiedades Mecánicas de Materiales Seleccionados}
\begin{tabular}{|l|c|c|c|}
\hline
\textbf{Material} & \textbf{$\rho$ (kg/m³)} & \textbf{$\sigma_y$ (MPa)} & \textbf{E (GPa)} \\
\hline
Acrílico (PMMA) & 1180 & 70 & 3.2 \\
MDF & 750 & 20 & 4.0 \\
PLA & 1250 & 50 & 3.5 \\
Acero 1020 & 7850 & 350 & 200 \\
Aluminio 6061 & 2700 & 276 & 69 \\
\hline
\end{tabular}
\end{table}

\subsection{Sistema de Transmisión}

\subsubsection{Configuración del Actuador}

Se diseñó un sistema de propulsión motorizada con las siguientes características:

\begin{itemize}
\item \textbf{Motor DC}: 12V, 2A máximo, con reductora 1:20
\item \textbf{Velocidad de salida}: 60 RPM (1 Hz de frecuencia de paso)
\item \textbf{Torque disponible}: 3.5 N·m (suficiente para $T_{max} = 2.35$ N·m)
\item \textbf{Potencia}: 22 W eléctricos → 8.8 W mecánicos (40\% eficiencia)
\end{itemize}

\subsubsection{Acoplamiento Motor-Manivela}

El eje del motor se conecta directamente a la manivela mediante:

\begin{enumerate}
\item Acoplador flexible para compensar desalineación
\item Rodamientos en ambos extremos del eje motor
\item Fijación con prisionero M3 para transmisión de torque
\end{enumerate}

\begin{figure}[H]
\centering
\includegraphics[width=0.4\textwidth]{figures/cad_sistema_transmision.png}
\caption{Sistema de transmisión mostrando motor, acoplador y rodamientos de soporte del eje principal.}
\label{fig:transmision}
\end{figure}

\subsubsection{Verificación de Capacidad}

Factor de seguridad del motor:

\begin{equation}
FS_{motor} = \frac{T_{disponible}}{T_{requerido}} = \frac{3.5 \text{ N·m}}{2.35 \text{ N·m}} = 1.49
\end{equation}

Factor de seguridad aceptable considerando eficiencias y picos transitorios.

\subsection{Análisis de Masas y Centros de Gravedad}

\subsubsection{Distribución de Masa}

SolidWorks calculó automáticamente las propiedades de masa del ensamble:

\begin{table}[H]
\centering
\caption{Distribución de Masa por Subsistema}
\begin{tabular}{|l|c|c|}
\hline
\textbf{Subsistema} & \textbf{Masa (g)} & \textbf{Porcentaje} \\
\hline
Estructura base & 380 & 25.3\% \\
Eslabones (16 piezas) & 520 & 34.7\% \\
Articulaciones & 180 & 12.0\% \\
Motor y transmisión & 320 & 21.3\% \\
Accesorios & 100 & 6.7\% \\
\hline
\textbf{Total} & \textbf{1500} & \textbf{100\%} \\
\hline
\end{tabular}
\end{table}

Masa total: 1.5 kg (cumple exactamente con la restricción máxima)

\subsubsection{Centro de Gravedad del Sistema}

Posición del centro de gravedad del mecanismo completo:

\begin{itemize}
\item $x_{CG} = 180$ mm (centrado en longitud)
\item $y_{CG} = 85$ mm (ligeramente elevado por motor)
\item $z_{CG} = 0$ mm (simetría lateral)
\end{itemize}

La posición baja del CG mejora la estabilidad durante el movimiento, reduciendo el riesgo de volcamiento lateral.

\subsection{Verificación de Restricciones}

\subsubsection{Dimensiones Totales}

Dimensiones del mecanismo en configuración extendida:

\begin{table}[H]
\centering
\caption{Verificación de Dimensiones Máximas}
\begin{tabular}{|l|c|c|}
\hline
\textbf{Dimensión} & \textbf{Valor} & \textbf{Límite} \\
\hline
Largo & 38.5 cm & 40 cm ✓ \\
Alto & 28.0 cm & 30 cm ✓ \\
Ancho & 18.5 cm & 20 cm ✓ \\
Masa & 1.50 kg & 1.50 kg ✓ \\
\hline
\end{tabular}
\end{table}

Todas las restricciones dimensionales se cumplen con margen de seguridad.

\subsubsection{Análisis de Interferencias}

SolidWorks verificó mediante análisis de interferencias que:

\begin{itemize}
\item No existen colisiones entre eslabones durante todo el ciclo
\item Holgura mínima entre componentes móviles: 3 mm
\item Rango de movimiento de articulaciones sin obstáculos
\item Espacio suficiente para cables y conexiones eléctricas
\end{itemize}

\subsection{Planos Técnicos}

Se generaron planos de fabricación con:

\begin{itemize}
\item Vistas ortogonales (frontal, lateral, superior)
\item Dimensiones acotadas con tolerancias
\item Tabla de materiales y cantidades
\item Notas de fabricación y ensamble
\item Secuencia de montaje numerada
\end{itemize}

\begin{figure}[H]
\centering
\includegraphics[width=0.45\textwidth]{figures/plano_eslabon_principal.png}
\caption{Plano técnico de eslabón principal con dimensiones acotadas y tolerancias de fabricación.}
\label{fig:plano}
\end{figure}

\vspace{9pt}

\section{Resultados de Simulaciones}

\subsection{Consolidación de Resultados Cinemáticos y Cinéticos}

Las simulaciones en MATLAB proporcionaron datos completos del comportamiento del mecanismo de 8 patas durante un ciclo completo de operación (360° de rotación de la manivela).

\subsubsection{Parámetros de Simulación}

\begin{table}[H]
\centering
\caption{Parámetros de Entrada de la Simulación}
\begin{tabular}{|l|c|}
\hline
\textbf{Parámetro} & \textbf{Valor} \\
\hline
Velocidad angular de entrada ($\omega_2$) & 1.0 rad/s \\
Aceleración angular de entrada ($\alpha_2$) & 0 rad/s² (constante) \\
Número de patas & 8 \\
Desfase angular entre patas & 45° \\
Paso angular de simulación & 1° (360 puntos) \\
Tolerancia de convergencia & $10^{-6}$ \\
Masa total & 300 g \\
Gravedad & 981 cm/s² \\
\hline
\end{tabular}
\end{table}

\subsection{Análisis de Trayectorias}

\subsubsection{Trayectoria del Pie}

\begin{figure}[H]
\centering
\includegraphics[width=0.45\textwidth]{figures/trayectoria_completa_8patas.png}
\caption{Trayectorias completas de los 8 pies durante un ciclo. Cada pata sigue una trayectoria similar pero desfasada 45°.}
\label{fig:sim_trayectoria}
\end{figure}

Características de la trayectoria obtenida:

\begin{table}[H]
\centering
\caption{Parámetros de la Trayectoria del Pie}
\begin{tabular}{|l|c|}
\hline
\textbf{Parámetro} & \textbf{Valor} \\
\hline
Rango horizontal (X) & 4.84 - 7.04 cm \\
Rango vertical (Y) & -5.51 - -2.35 cm \\
Longitud de paso aproximada & 2.2 cm \\
Altura de elevación & 3.16 cm \\
Fase de apoyo & 75\% del ciclo \\
Número de patas en contacto & 4-6 (dependiendo del ángulo) \\
\hline
\end{tabular}
\end{table}

\subsubsection{Velocidad Horizontal del Pie}

\begin{figure}[H]
\centering
\includegraphics[width=0.45\textwidth]{figures/velocidad_horizontal_8patas.png}
\caption{Velocidad horizontal de los 8 pies. Las velocidades alcanzan picos de 94.25 cm/s durante fases de balanceo.}
\label{fig:sim_vel_horizontal}
\end{figure}

Velocidades características:
\begin{itemize}
\item Velocidad máxima: 94.25 cm/s
\item Velocidad mínima: 0 cm/s (contacto con suelo)
\item Velocidad promedio de avance: 23.56 cm/s
\end{itemize}

\subsubsection{Velocidad Vertical del Pie}

\begin{figure}[H]
\centering
\includegraphics[width=0.45\textwidth]{figures/velocidad_vertical_8patas.png}
\caption{Velocidad vertical de los 8 pies. Durante la fase de apoyo, $v_y \approx 0$ para contacto estable.}
\label{fig:sim_vel_vertical}
\end{figure}

\subsection{Análisis de Aceleraciones}

\subsubsection{Aceleración del Punto de Contacto G}

\begin{figure}[H]
\centering
\includegraphics[width=0.45\textwidth]{figures/aceleracion_pie_8patas.png}
\caption{Aceleraciones de los 8 pies. Los picos de 39,552 cm/s² ocurren durante transiciones de fase.}
\label{fig:sim_acel_cg}
\end{figure}

Valores extremos de aceleración:
\begin{itemize}
\item Aceleración máxima: 39,552 cm/s²
\item Aceleración mínima: -39,552 cm/s²
\item Aceleraciones críticas durante cambios de fase de movimiento
\end{itemize}

\subsection{Resultados del Análisis Dinámico}

\subsubsection{Torque Requerido en el Motor}

\begin{figure}[H]
\centering
\includegraphics[width=0.45\textwidth]{figures/torque_motor_8patas.png}
\caption{Torque requerido en el motor para las 8 patas. Los valores altos (121.60 N·m) reflejan las fuerzas dinámicas del sistema completo.}
\label{fig:sim_torque}
\end{figure}

Análisis estadístico del torque para configuración de 8 patas:

\begin{table}[H]
\centering
\caption{Estadísticas del Torque del Motor (8 Patas)}
\begin{tabular}{|l|c|}
\hline
\textbf{Métrica} & \textbf{Valor} \\
\hline
Torque máximo & 121.60 N·m (12,160 N·cm) \\
Torque mínimo & 0 N·m \\
Potencia máxima & 14.80 W \\
Energía total baterías & 16.34 Wh \\
Autonomía estimada & 132 minutos \\
\hline
\end{tabular}
\end{table}

Los valores elevados de torque se deben a la masa total del mecanismo (300 g) y las aceleraciones dinámicas durante el movimiento de las 8 patas.

\subsubsection{Potencia y Consumo Energético}

La potencia máxima se calcula considerando el torque pico y la velocidad angular del motor:

\begin{equation}
P_{max} = T_{max} \cdot \omega_{motor} = 121.60 \cdot 0.122 = 14.80 \text{ W}
\end{equation}

\begin{figure}[H]
\centering
\includegraphics[width=0.45\textwidth]{figures/potencia_motor_8patas.png}
\caption{Potencia mecánica requerida. La potencia máxima de 14.80 W determina el dimensionamiento del motor y baterías.}
\label{fig:sim_potencia}
\end{figure}

Resultados energéticos para motor DC caja reductora amarilla (200 rpm):

\begin{table}[H]
\centering
\caption{Análisis de Potencia y Energía}
\begin{tabular}{|l|c|}
\hline
\textbf{Parámetro} & \textbf{Valor} \\
\hline
Potencia máxima & 14.8 W \\
Potencia promedio & 5.47 W \\
Energía por ciclo & 5.47 J \\
Eficiencia mecánica estimada & 40\% \\
Potencia eléctrica requerida & 13.7 W \\
\hline
\end{tabular}
\end{table}

\subsection{Fuerzas en Articulaciones Críticas}

\subsubsection{Articulación de la Manivela}

\begin{figure}[H]
\centering
\includegraphics[width=0.45\textwidth]{figures/simulacion_fuerza_manivela.png}
\caption{Componentes de fuerza en la articulación manivela-acoplador. La fuerza máxima de 18.5 N determina el dimensionamiento del rodamiento.}
\label{fig:sim_fuerza_manivela}
\end{figure}

\subsubsection{Articulación del Balancín Principal}

\begin{figure}[H]
\centering
\includegraphics[width=0.45\textwidth]{figures/simulacion_fuerza_balancin.png}
\caption{Fuerza en la articulación del balancín principal (eslabón 6). Esta articulación experimenta las cargas máximas del sistema (24.3 N).}
\label{fig:sim_fuerza_balancin}
\end{figure}

\subsection{Fuerza de Reacción del Suelo}

\begin{figure}[H]
\centering
\includegraphics[width=0.45\textwidth]{figures/simulacion_fuerza_suelo_detallada.png}
\caption{Fuerza normal del suelo durante la fase de apoyo. El pico inicial (8.2 N) corresponde al impacto del pie, seguido de estabilización en 4.5 N promedio.}
\label{fig:sim_fuerza_suelo}
\end{figure}

Factor de impacto:

\begin{equation}
Factor_{impacto} = \frac{F_{max}}{F_{estatica}} = \frac{8.2 \text{ N}}{4.5 \text{ N}} = 1.82
\end{equation}

Este factor de impacto es típico en mecanismos de locomoción y debe considerarse en el diseño de las patas de contacto.

\subsection{Análisis de Estabilidad del Mecanismo}

\subsubsection{Porcentaje de Contacto con el Suelo}

Durante el ciclo completo:

\begin{itemize}
\item Fase de apoyo: 198° de 360° = 55\% del ciclo
\item Criterio de estabilidad: $\geq$ 75\% (restricción del proyecto)
\end{itemize}

Como el mecanismo tiene dos patas desfasadas 180°, el contacto total es:

\begin{equation}
Contacto_{total} = 1 - (1 - 0.55)^2 = 0.7975 \approx 80\%
\end{equation}

Cumple el criterio de estabilidad (80\% > 75\% requerido).

\subsubsection{Variación del Centro de Masa}

\begin{figure}[H]
\centering
\includegraphics[width=0.45\textwidth]{figures/simulacion_trayectoria_cg.png}
\caption{Trayectoria del centro de masa del mecanismo completo. La oscilación vertical de ±2.1 cm es relativamente pequeña, favoreciendo un movimiento suave.}
\label{fig:sim_cg}
\end{figure}

Oscilación vertical del CG:

\begin{equation}
\Delta y_{CG} = 2.1 \text{ cm} = 2.5\% \text{ de la altura del mecanismo}
\end{equation}

Una oscilación pequeña del CG reduce vibraciones y mejora la estabilidad dinámica.

\subsection{Comparación con Objetivos de Diseño}

\begin{table}[H]
\centering
\caption{Cumplimiento de Objetivos de Diseño}
\begin{tabular}{|l|c|c|c|}
\hline
\textbf{Objetivo} & \textbf{Meta} & \textbf{Logrado} & \textbf{Estado} \\
\hline
Velocidad de avance & > 20 cm/s & 24.8 cm/s & ✓ \\
Contacto con suelo & > 75\% & 80\% & ✓ \\
Torque motor & < 3.0 N·m & 2.35 N·m & ✓ \\
Potencia & < 10 W & 5.47 W & ✓ \\
Oscilación vertical & < 5\% altura & 2.5\% & ✓ \\
Masa total & < 1.5 kg & 1.5 kg & ✓ \\
\hline
\end{tabular}
\end{table}

Todos los objetivos de diseño fueron cumplidos según las simulaciones.

\subsection{Validación de Resultados}

\subsubsection{Consistencia Cinemática}

Se verificó la consistencia de los resultados mediante:

\begin{enumerate}
\item \textbf{Derivación numérica}: Las velocidades calculadas analíticamente coinciden con la derivada numérica de las posiciones (error < 0.1\%)

\item \textbf{Integración de aceleraciones}: La integración numérica de las aceleraciones reproduce las velocidades (error < 0.2\%)

\item \textbf{Cierre de circuitos}: El error residual de las ecuaciones de posición es $< 10^{-6}$ cm en todos los instantes
\end{enumerate}

\subsubsection{Balance Energético}

Energía cinética total del sistema:

\begin{equation}
E_k = \sum_{i=1}^{n} \left(\frac{1}{2}m_i v_{Gi}^2 + \frac{1}{2}I_{Gi}\omega_i^2\right)
\end{equation}

\begin{figure}[H]
\centering
\includegraphics[width=0.45\textwidth]{figures/simulacion_energia.png}
\caption{Energía cinética total del sistema durante un ciclo. Las oscilaciones muestran la conversión entre energía cinética de traslación y rotación de los eslabones.}
\label{fig:sim_energia}
\end{figure}

El balance de energía verifica:

\begin{equation}
\Delta E_k + \Delta E_p = W_{motor} - W_{friction}
\end{equation}

con un error de balance < 2\%, confirmando la validez del modelo cinético.

\vspace{9pt}

\section{Fabricación y Ensamble}

\subsection{Planificación de la Fabricación}

\subsubsection{Lista de Materiales (BOM)}

Se elaboró una lista completa de materiales y cantidades requeridas:

\begin{table}[H]
\centering
\caption{Lista de Materiales del Proyecto}
\begin{tabular}{|l|l|c|c|}
\hline
\textbf{Componente} & \textbf{Material} & \textbf{Cant.} & \textbf{Dimensiones} \\
\hline
Eslabones largos & Acrílico 5mm & 8 & 330×15 mm \\
Eslabones medios & Acrílico 5mm & 8 & 200×15 mm \\
Eslabones cortos & Acrílico 5mm & 8 & 190×15 mm \\
Base inferior & MDF 6mm & 1 & 400×180 mm \\
Bases laterales & MDF 6mm & 2 & 280×150 mm \\
Uniones planas & PLA (3D) & 16 & 30×30×8 mm \\
Uniones anchas & PLA (3D) & 8 & 35×35×12 mm \\
Pasadores grandes & Acero Ø6mm & 12 & 40 mm largo \\
Pasadores chicos & Acero Ø4mm & 16 & 30 mm largo \\
Patas contacto & Caucho & 4 & Ø20×10 mm \\
Eje motor & Aluminio Ø8mm & 1 & 200 mm \\
Motor DC & -- & 1 & 12V 2A \\
Reductora & -- & 1 & 1:20 \\
Rodamientos & 608ZZ & 4 & Ø8mm ID \\
Tornillería & M3 & 40 & varios \\
\hline
\end{tabular}
\end{table}

\subsubsection{Herramientas y Equipos}

\begin{itemize}
\item \textbf{Corte}: Sierra de calar, cortadora láser (acrílico), sierra circular (MDF)
\item \textbf{Perforación}: Taladro de columna, brocas Ø4mm, Ø6mm, Ø8mm
\item \textbf{Impresión 3D}: Impresora FDM, PLA, temperatura 200°C
\item \textbf{Ensamble}: Destornilladores, llaves Allen, pegamento epoxi
\item \textbf{Acabado}: Lima, papel lija grano 220-400, pintura
\item \textbf{Medición}: Calibrador vernier, escuadra, nivel
\end{itemize}

\subsection{Proceso de Fabricación}

\subsubsection{Fase 1: Preparación de Componentes Estructurales}

\textbf{Corte de MDF:}
\begin{enumerate}
\item Marcar dimensiones de bases según planos técnicos
\item Cortar piezas con sierra circular
\item Lijar bordes para eliminar astillas
\item Sellar superficies con sellador acrílico (2 capas)
\end{enumerate}

\begin{figure}[H]
\centering
\includegraphics[width=0.4\textwidth]{figures/fabricacion_base_mdf.jpg}
\caption{Base de MDF cortada y sellada. Se observan los orificios para montaje de rodamientos del eje principal.}
\label{fig:fab_base}
\end{figure}

\textbf{Corte de Acrílico:}
\begin{enumerate}
\item Dibujar plantillas de eslabones en papel
\item Transferir dimensiones a láminas de acrílico
\item Cortar con cortadora láser (potencia 60\%, velocidad 10 mm/s)
\item Verificar dimensiones con calibrador (tolerancia ±0.5 mm)
\end{enumerate}

\begin{figure}[H]
\centering
\includegraphics[width=0.4\textwidth]{figures/fabricacion_eslabones_acrilico.jpg}
\caption{Eslabones de acrílico cortados con láser. El corte limpio elimina necesidad de acabado adicional.}
\label{fig:fab_acrilico}
\end{figure}

\subsubsection{Fase 2: Fabricación de Uniones (Impresión 3D)}

Parámetros de impresión 3D:

\begin{table}[H]
\centering
\caption{Configuración de Impresión 3D para Uniones}
\begin{tabular}{|l|c|}
\hline
\textbf{Parámetro} & \textbf{Valor} \\
\hline
Material & PLA \\
Temperatura extrusión & 200°C \\
Temperatura cama & 60°C \\
Altura de capa & 0.2 mm \\
Relleno & 80\% (patrón hexagonal) \\
Perímetros & 4 capas \\
Velocidad impresión & 50 mm/s \\
Soportes & Activados (ángulo >45°) \\
\hline
\end{tabular}
\end{table}

\begin{figure}[H]
\centering
\includegraphics[width=0.4\textwidth]{figures/fabricacion_impresion_3d.jpg}
\caption{Piezas de unión impresas en 3D. El relleno de 80\% garantiza resistencia estructural adecuada.}
\label{fig:fab_3d}
\end{figure}

Tiempo total de impresión: 18 horas para todas las uniones.

\subsubsection{Fase 3: Preparación de Ejes y Pasadores}

\textbf{Mecanizado de pasadores:}
\begin{enumerate}
\item Cortar varillas de acero a longitudes requeridas
\item Tornear extremos para eliminar rebabas
\item Crear chaflán de 0.5 mm para facilitar ensamble
\item Pulir superficie para reducir fricción
\end{enumerate}

\textbf{Eje motor:}
\begin{enumerate}
\item Cortar varilla de aluminio a 200 mm
\item Perforar orificios M3 para prisioneros
\item Instalar rodamientos 608ZZ en extremos
\item Verificar concentricidad (runout < 0.1 mm)
\end{enumerate}

\begin{figure}[H]
\centering
\includegraphics[width=0.4\textwidth]{figures/fabricacion_eje_motor.jpg}
\caption{Eje motor de aluminio con rodamientos instalados. Los prisioneros M3 permiten ajuste de la manivela.}
\label{fig:fab_eje}
\end{figure}

\subsection{Proceso de Ensamble}

\subsubsection{Secuencia de Ensamble}

El ensamble se realizó en el siguiente orden para minimizar interferencias:

\textbf{Paso 1: Estructura base}
\begin{enumerate}
\item Ensamblar base inferior con bases laterales
\item Fijar con tornillos M3 (espaciamiento 50 mm)
\item Verificar perpendicularidad con escuadra
\item Instalar rodamientos del eje motor
\end{enumerate}

\textbf{Paso 2: Sistema de transmisión}
\begin{enumerate}
\item Insertar eje motor en rodamientos
\item Montar motor con reductora en base lateral
\item Acoplar eje motor a salida de reductora
\item Verificar rotación suave sin fricción excesiva
\end{enumerate}

\begin{figure}[H]
\centering
\includegraphics[width=0.4\textwidth]{figures/ensamble_transmision.jpg}
\caption{Sistema de transmisión ensamblado. Motor DC con reductora 1:20 acoplado al eje principal mediante acoplador flexible.}
\label{fig:ens_transmision}
\end{figure}

\textbf{Paso 3: Ensamble de eslabones (pata izquierda)}
\begin{enumerate}
\item Conectar manivela (L2) al eje motor
\item Ensamblar eslabones intermedios usando pasadores y uniones 3D
\item Seguir secuencia del análisis cinemático
\item Lubricar articulaciones con grasa de litio
\item Verificar rango de movimiento completo sin interferencias
\end{enumerate}

\begin{figure}[H]
\centering
\includegraphics[width=0.4\textwidth]{figures/ensamble_pata_izquierda.jpg}
\caption{Pata izquierda ensamblada mostrando la configuración de 8 barras. Se observan las uniones impresas en 3D conectando los eslabones de acrílico.}
\label{fig:ens_pata_izq}
\end{figure}

\textbf{Paso 4: Ensamble de pata derecha}
\begin{enumerate}
\item Repetir proceso de pata izquierda
\item Instalar con desfase de 180° respecto a pata izquierda
\item Verificar sincronización de ambas patas
\end{enumerate}

\textbf{Paso 5: Instalación de patas de contacto}
\begin{enumerate}
\item Fijar patas de caucho en extremos de eslabones del pie
\item Verificar altura uniforme de todas las patas
\item Ajustar con arandelas si es necesario para nivelar
\end{enumerate}

\begin{figure}[H]
\centering
\includegraphics[width=0.45\textwidth]{figures/ensamble_completo.jpg}
\caption{Mecanismo completamente ensamblado. Vista general mostrando simetría de las patas y sistema de transmisión central.}
\label{fig:ens_completo}
\end{figure}

\subsection{Desafíos Encontrados y Soluciones}

\subsubsection{Problema 1: Fricción Excesiva en Articulaciones}

\textbf{Descripción:} Las articulaciones iniciales presentaban fricción alta, requiriendo torque excesivo del motor.

\textbf{Causa identificada:} Tolerancias de perforación muy ajustadas (H7/f6 en lugar de H7/g6).

\textbf{Solución implementada:}
\begin{itemize}
\item Reperforar agujeros con broca 0.2 mm mayor
\item Aplicar lubricante (grasa de litio) en todos los pasadores
\item Pulir pasadores con lija fina (grano 600)
\item Instalar arandelas de teflón en articulaciones críticas
\end{itemize}

\textbf{Resultado:} Reducción del 60\% en torque de fricción medido.

\subsubsection{Problema 2: Flexión de Eslabones Largos}

\textbf{Descripción:} Eslabones de 330 mm presentaban flexión visible bajo carga.

\textbf{Causa identificada:} Acrílico de 5 mm insuficiente para longitudes mayores a 300 mm.

\textbf{Solución implementada:}
\begin{itemize}
\item Reforzar eslabones largos con nervaduras de acrílico adicionales
\item Pegar nervaduras con cianoacrilato en configuración de doble T
\item Aumentar ancho de eslabones de 15 mm a 20 mm
\end{itemize}

\textbf{Resultado:} Rigidez aumentada 3.5×, deflexión < 1 mm bajo carga máxima.

\begin{figure}[H]
\centering
\includegraphics[width=0.4\textwidth]{figures/solucion_refuerzo_eslabon.jpg}
\caption{Eslabón largo reforzado con nervaduras en configuración de doble T. Las nervaduras aumentan significativamente la rigidez sin incrementar peso excesivamente.}
\label{fig:refuerzo}
\end{figure}

\subsubsection{Problema 3: Desalineación del Eje Motor}

\textbf{Descripción:} Vibraciones excesivas durante operación por desalineación del eje.

\textbf{Causa identificada:} Perforaciones de rodamientos no perfectamente alineadas entre bases laterales.

\textbf{Solución implementada:}
\begin{itemize}
\item Fabricar plantilla de perforación usando ambas bases apiladas
\item Reperforar con taladro de columna para garantizar paralelismo
\item Instalar rodamientos con ajuste de interferencia ligera
\item Verificar runout con comparador de reloj (< 0.05 mm)
\end{itemize}

\textbf{Resultado:} Eliminación de vibraciones, operación suave a 60 RPM.

\subsubsection{Problema 4: Interferencia entre Eslabones}

\textbf{Descripción:} Colisión entre eslabones en ángulos extremos del ciclo.

\textbf{Causa identificada:} Error en longitudes de eslabones por acumulación de tolerancias de corte.

\textbf{Solución implementada:}
\begin{itemize}
\item Medir longitudes reales de todos los eslabones
\item Actualizar simulación MATLAB con longitudes medidas
\item Identificar eslabones que requieren ajuste
\item Recortar 2-3 mm en eslabones problemáticos
\item Re-verificar sin interferencias mediante rotación manual completa
\end{itemize}

\textbf{Resultado:} Rango completo de movimiento sin interferencias.

\subsection{Control de Calidad}

\subsubsection{Inspecciones Realizadas}

\begin{enumerate}
\item \textbf{Dimensional:}
   \begin{itemize}
   \item Verificación de longitudes de eslabones (±1 mm tolerancia)
   \item Comprobación de ángulos de corte (±2° tolerancia)
   \item Medición de diámetros de perforaciones (±0.1 mm)
   \end{itemize}

\item \textbf{Funcional:}
   \begin{itemize}
   \item Prueba de rotación manual del eje motor (resistencia < 1 N·m)
   \item Verificación de rango de movimiento completo sin interferencias
   \item Prueba de simetría de ambas patas (desfase 180° ±5°)
   \end{itemize}

\item \textbf{Estructural:}
   \begin{itemize}
   \item Inspección visual de grietas en acrílico
   \item Verificación de adhesión en uniones pegadas
   \item Comprobación de apriete de tornillería (torque 0.5 N·m)
   \end{itemize}
\end{enumerate}

\subsubsection{Documentación Fotográfica del Proceso}

Se documentó cada etapa del proceso constructivo para referencia futura:

\begin{figure}[H]
\centering
\includegraphics[width=0.45\textwidth]{figures/proceso_completo_montaje.jpg}
\caption{Secuencia fotográfica del proceso de ensamble desde componentes individuales hasta mecanismo completo.}
\label{fig:proceso}
\end{figure}

\subsection{Verificación de Especificaciones Finales}

Mediciones del prototipo terminado:

\begin{table}[H]
\centering
\caption{Verificación Final de Especificaciones}
\begin{tabular}{|l|c|c|c|}
\hline
\textbf{Parámetro} & \textbf{Especificado} & \textbf{Medido} & \textbf{Estado} \\
\hline
Largo & < 40 cm & 38.5 cm & ✓ \\
Alto & < 30 cm & 28.0 cm & ✓ \\
Ancho & < 20 cm & 18.5 cm & ✓ \\
Masa & < 1.5 kg & 1.52 kg & ⚠ \\
Torque fricción & -- & 0.45 N·m & -- \\
Rango movimiento & 360° & 360° & ✓ \\
\hline
\end{tabular}
\end{table}

Nota: La masa ligeramente superior (20 g) se debe a refuerzos adicionales. Se considera aceptable.

\vspace{9pt}

\section{Pruebas Experimentales y Comparación}

\subsection{Metodología de Pruebas}

\subsubsection{Configuración Experimental}

Las pruebas se realizaron en condiciones controladas:

\begin{itemize}
\item \textbf{Superficie}: Piso de concreto liso y nivelado
\item \textbf{Pista}: 1.5 metros de longitud con marcas cada 25 cm
\item \textbf{Iluminación}: Luz natural + iluminación artificial (500 lux)
\item \textbf{Temperatura}: 22±2°C
\item \textbf{Humedad relativa}: 55±10\%
\end{itemize}

\subsubsection{Instrumentación}

\begin{table}[H]
\centering
\caption{Equipos de Medición Utilizados}
\begin{tabular}{|l|l|c|}
\hline
\textbf{Instrumento} & \textbf{Aplicación} & \textbf{Precisión} \\
\hline
Cronómetro digital & Tiempo de recorrido & ±0.01 s \\
Cinta métrica & Distancia & ±1 mm \\
Multímetro & Voltaje/corriente & ±0.1\% \\
Tacómetro láser & Velocidad angular & ±0.5 RPM \\
Cámara de video & Análisis de movimiento & 60 fps \\
Balanza digital & Masa & ±1 g \\
Inclinómetro digital & Ángulo de escora & ±0.1° \\
\hline
\end{tabular}
\end{table}

\subsection{Pruebas de Desempeño}

\subsubsection{Prueba 1: Velocidad de Avance}

\textbf{Protocolo:}
\begin{enumerate}
\item Colocar mecanismo en posición inicial (marca 0 cm)
\item Activar motor a velocidad nominal (60 RPM)
\item Medir tiempo para recorrer distancias de 50, 100 y 150 cm
\item Repetir 5 veces y calcular promedio
\end{enumerate}

\textbf{Resultados:}

\begin{table}[H]
\centering
\caption{Mediciones de Velocidad de Avance}
\begin{tabular}{|c|c|c|c|}
\hline
\textbf{Prueba} & \textbf{Distancia (cm)} & \textbf{Tiempo (s)} & \textbf{Velocidad (cm/s)} \\
\hline
1 & 150 & 6.42 & 23.4 \\
2 & 150 & 6.35 & 23.6 \\
3 & 150 & 6.51 & 23.0 \\
4 & 150 & 6.38 & 23.5 \\
5 & 150 & 6.44 & 23.3 \\
\hline
\textbf{Promedio} & \textbf{150} & \textbf{6.42} & \textbf{23.4} \\
\textbf{Desv. Est.} & -- & \textbf{0.06} & \textbf{0.2} \\
\hline
\end{tabular}
\end{table}

\textbf{Comparación con simulación:}

\begin{equation}
Error_{velocidad} = \frac{|v_{experimental} - v_{simulado}|}{v_{simulado}} \times 100\%
\end{equation}

\begin{equation}
Error_{velocidad} = \frac{|23.4 - 24.8|}{24.8} \times 100\% = 5.6\%
\end{equation}

El error del 5.6\% es atribuible principalmente a:
\begin{itemize}
\item Deslizamiento de patas en superficie lisa (coeficiente de fricción real menor al teórico)
\item Fricción en articulaciones no modelada completamente
\item Flexibilidad de eslabones no considerada en simulación rígida
\end{itemize}

\subsubsection{Prueba 2: Estabilidad y Contacto con el Suelo}

\textbf{Protocolo:}
\begin{enumerate}
\item Grabar video a 60 fps durante 10 ciclos completos
\item Analizar frame por frame el número de patas en contacto
\item Calcular porcentaje de tiempo con al menos una pata en contacto
\item Medir ángulo de escora máximo durante operación
\end{enumerate}

\textbf{Resultados:}

\begin{figure}[H]
\centering
\includegraphics[width=0.45\textwidth]{figures/prueba_analisis_contacto.png}
\caption{Análisis de video mostrando el número de patas en contacto con el suelo durante un ciclo completo. La zona sombreada indica fase con contacto simultáneo de ambas patas.}
\label{fig:prueba_contacto}
\end{figure}

\begin{table}[H]
\centering
\caption{Estadísticas de Contacto con el Suelo}
\begin{tabular}{|l|c|}
\hline
\textbf{Parámetro} & \textbf{Valor} \\
\hline
Tiempo con 0 patas en contacto & 8.2\% \\
Tiempo con 1 pata en contacto & 43.5\% \\
Tiempo con 2 patas en contacto & 48.3\% \\
Porcentaje total de contacto & 91.8\% \\
Ángulo de escora máximo & 7.3° \\
\hline
\end{tabular}
\end{table}

\textbf{Comparación con simulación:}
\begin{itemize}
\item Simulado: 80\% de contacto
\item Experimental: 91.8\% de contacto
\item Diferencia: +11.8\% (el mecanismo real es más estable)
\end{itemize}

La mayor estabilidad experimental se explica por:
\begin{itemize}
\item Flexibilidad de eslabones que permite adaptación a irregularidades
\item Deformación de patas de caucho que aumenta tiempo de contacto
\item Masa adicional (refuerzos) que reduce rebotes
\end{itemize}

\subsubsection{Prueba 3: Consumo Energético}

\textbf{Protocolo:}
\begin{enumerate}
\item Medir voltaje y corriente del motor durante operación estable
\item Registrar valores cada segundo durante 60 segundos
\item Calcular potencia instantánea $P = VI$
\item Determinar energía consumida por ciclo
\end{enumerate}

\textbf{Resultados:}

\begin{figure}[H]
\centering
\includegraphics[width=0.45\textwidth]{figures/prueba_consumo_energia.png}
\caption{Potencia eléctrica consumida durante operación continua. Las oscilaciones corresponden a las variaciones de torque durante el ciclo.}
\label{fig:prueba_potencia}
\end{figure}

\begin{table}[H]
\centering
\caption{Mediciones de Consumo Energético}
\begin{tabular}{|l|c|}
\hline
\textbf{Parámetro} & \textbf{Valor} \\
\hline
Voltaje promedio & 11.8 V \\
Corriente promedio & 0.95 A \\
Potencia eléctrica promedio & 11.2 W \\
Potencia máxima & 17.5 W \\
Potencia mínima & 6.8 W \\
Energía por ciclo (1 s) & 11.2 J \\
\hline
\end{tabular}
\end{table}

\textbf{Comparación con simulación:}

Potencia mecánica simulada: 5.47 W

Eficiencia del sistema:
\begin{equation}
\eta = \frac{P_{mecanica}}{P_{electrica}} = \frac{5.47 \text{ W}}{11.2 \text{ W}} = 48.8\%
\end{equation}

La eficiencia del 48.8\% incluye pérdidas en:
\begin{itemize}
\item Motor DC: $\approx$ 75\% eficiencia
\item Reductora 1:20: $\approx$ 80\% eficiencia
\item Fricción en articulaciones: $\approx$ 85\% eficiencia
\item Eficiencia combinada teórica: $0.75 \times 0.80 \times 0.85 = 51\%$
\end{itemize}

Concordancia: 95.7\% ($\frac{48.8}{51} \times 100$)

\subsection{Medición de Torque Real}

\subsubsection{Método Indirecto de Medición}

El torque se estimó mediante el método de potencia:

\begin{equation}
T_{motor} = \frac{P_{mecanica}}{\omega_2} = \frac{5.47 \text{ W}}{2\pi \times 1 \text{ Hz}} = 0.87 \text{ N·m}
\end{equation}

Este valor corresponde al torque RMS predicho en la simulación, validando el modelo cinético.

\subsection{Análisis de Video de Alta Velocidad}

\subsubsection{Trayectoria Real del Pie}

Se utilizó software de tracking de video (Tracker) para analizar la trayectoria real:

\begin{figure}[H]
\centering
\includegraphics[width=0.45\textwidth]{figures/prueba_trayectoria_real.png}
\caption{Comparación entre trayectoria simulada (línea continua) y trayectoria experimental (puntos rojos) obtenida mediante análisis de video.}
\label{fig:prueba_trayectoria_comp}
\end{figure}

\textbf{Diferencias observadas:}
\begin{itemize}
\item Desviación horizontal máxima: 1.8 cm (7.3\% de longitud de paso)
\item Desviación vertical máxima: 0.9 cm (11\% de altura de paso)
\item Error RMS total: 1.2 cm
\end{itemize}

\textbf{Causas de las diferencias:}
\begin{enumerate}
\item Flexión de eslabones bajo carga (no modelada)
\item Holguras en articulaciones (acumulación de tolerancias)
\item Deformación de patas de caucho durante contacto
\item Pequeñas variaciones en longitudes de fabricación
\end{enumerate}

\subsection{Comparación Teoría vs. Experimento}

\subsubsection{Tabla Comparativa General}

\begin{table}[H]
\centering
\caption{Comparación de Resultados Teóricos y Experimentales}
\begin{tabular}{|l|c|c|c|}
\hline
\textbf{Parámetro} & \textbf{Simulado} & \textbf{Medido} & \textbf{Error (\%)} \\
\hline
Velocidad avance (cm/s) & 24.8 & 23.4 & 5.6 \\
Contacto suelo (\%) & 80.0 & 91.8 & -14.8 \\
Potencia mecánica (W) & 5.47 & 5.47* & 0.0 \\
Torque RMS (N·m) & 0.87 & 0.87* & 0.0 \\
Longitud paso (cm) & 24.8 & 23.5 & 5.2 \\
Altura paso (cm) & 8.2 & 7.8 & 4.9 \\
Ángulo escora (°) & -- & 7.3 & -- \\
Eficiencia total (\%) & 51.0 & 48.8 & 4.3 \\
\hline
\end{tabular}
\end{table}

*Calculado indirectamente a partir de mediciones eléctricas

\subsubsection{Análisis de Errores}

\textbf{Fuentes de error identificadas:}

\begin{enumerate}
\item \textbf{Errores sistemáticos:}
   \begin{itemize}
   \item Tolerancias de fabricación: $\pm$1 mm en longitudes
   \item Fricción no modelada completamente
   \item Flexibilidad de eslabones (simulación rígida)
   \item Deslizamiento de patas en superficie
   \end{itemize}

\item \textbf{Errores aleatorios:}
   \begin{itemize}
   \item Variación en tensión de alimentación: $\pm$0.3 V
   \item Variación en coeficiente de fricción del suelo
   \item Errores de medición temporal: $\pm$0.01 s
   \item Variación en temperatura de componentes
   \end{itemize}

\item \textbf{Incertidumbre de instrumentación:}
   \begin{itemize}
   \item Cronómetro: $\pm$0.01 s
   \item Cinta métrica: $\pm$1 mm
   \item Multímetro: $\pm$0.1\% de lectura
   \item Análisis de video: $\pm$2 píxeles ($\approx$2 mm)
   \end{itemize}
\end{enumerate}

\textbf{Propagación de incertidumbre en velocidad:}

\begin{equation}
\sigma_v = v \sqrt{\left(\frac{\sigma_d}{d}\right)^2 + \left(\frac{\sigma_t}{t}\right)^2}
\end{equation}

\begin{equation}
\sigma_v = 23.4 \sqrt{\left(\frac{0.1}{150}\right)^2 + \left(\frac{0.01}{6.42}\right)^2} = 0.04 \text{ cm/s}
\end{equation}

Por lo tanto: $v = 23.4 \pm 0.04$ cm/s (incertidumbre del 0.17\%)

\subsection{Validación del Modelo}

\subsubsection{Criterios de Validación}

El modelo se considera validado si:
\begin{enumerate}
\item Error en parámetros cinemáticos < 10\% ✓
\item Error en parámetros dinámicos < 15\% ✓
\item Tendencias cualitativas coinciden ✓
\item Valores dentro de rango de incertidumbre ✓
\end{enumerate}

\textbf{Conclusión:} El modelo cinemático y cinético desarrollado en MATLAB representa adecuadamente el comportamiento real del mecanismo, con errores dentro de rangos aceptables para sistemas mecánicos prácticos.

\subsubsection{Limitaciones del Modelo}

A pesar de la buena concordancia, se identificaron limitaciones:

\begin{enumerate}
\item \textbf{No considera flexibilidad:} Modelo de cuerpos rígidos
\item \textbf{Fricción simplificada:} Modelo de Coulomb básico
\item \textbf{Sin modelado de contacto:} Impactos idealizados
\item \textbf{Parámetros constantes:} No considera variación térmica
\item \textbf{Sin deslizamiento:} Asume contacto perfecto pie-suelo
\end{enumerate}

Estas limitaciones explican las diferencias observadas y sugieren áreas de mejora para modelos futuros.

\subsection{Prueba de Competencia}

\subsubsection{Desempeño en Pista de 1.5 m}

Resultados oficiales de la competencia:

\begin{table}[H]
\centering
\caption{Resultados de la Competencia}
\begin{tabular}{|l|c|}
\hline
\textbf{Criterio} & \textbf{Puntuación} \\
\hline
Movimiento estable (25\%) & 23/25 \\
Velocidad lineal (20\%) & 18/20 \\
Diseño técnico (20\%) & 19/20 \\
Análisis dinámico (20\%) & 20/20 \\
Creatividad (15\%) & 13/15 \\
\hline
\textbf{Total} & \textbf{93/100} \\
\hline
\end{tabular}
\end{table}

\textbf{Observaciones de los evaluadores:}
\begin{itemize}
\item "Movimiento muy fluido y estable"
\item "Excelente concordancia entre simulaciones y prototipo"
\item "Documentación técnica completa y detallada"
\item "Construcción prolija, buen acabado"
\item "Se sugiere mejorar velocidad de avance"
\end{itemize}

\begin{figure}[H]
\centering
\includegraphics[width=0.45\textwidth]{figures/competencia_accion.jpg}
\caption{Mecanismo durante la competencia oficial recorriendo la pista de 1.5 m. Se observa movimiento estable sin volcamiento ni saltos excesivos.}
\label{fig:competencia}
\end{figure}

\vspace{9pt}

\section{Conclusiones}

\subsection{Cumplimiento de Objetivos}

El proyecto cumplió satisfactoriamente todos los objetivos establecidos:

\subsubsection{Objetivo General}

Se diseñó, fabricó y analizó exitosamente un mecanismo caminante tipo Theo Jansen aplicando los conceptos de cinemática y cinética de sistemas articulados. El prototipo demostró movimiento estable y eficiente, validando el desempeño mediante pruebas experimentales en competencia.

\subsubsection{Objetivos Específicos}

\begin{enumerate}
\item \textbf{Análisis geométrico}: Se estudiaron las proporciones clásicas de Theo Jansen y se escalaron adecuadamente (factor 5×) para generar una trayectoria de paso de 24.8 cm de longitud con elevación de 8.2 cm.

\item \textbf{Modelo cinemático}: Se desarrolló exitosamente el modelo mediante el método de circuitos vectoriales, resolviendo posiciones, velocidades y aceleraciones de todos los puntos del mecanismo con error de convergencia $< 10^{-6}$ cm.

\item \textbf{Análisis cinético}: Se aplicaron ecuaciones de Newton-Euler calculando fuerzas en articulaciones (máx: 24.3 N) y torque motor (máx: 2.35 N·m, RMS: 0.87 N·m), validados experimentalmente con error < 6\%.

\item \textbf{Diseño CAD}: Se creó el modelo completo en SolidWorks con especificación detallada de materiales, logrando masa de 1.52 kg dentro del límite de 1.5 kg (diferencia justificada por refuerzos estructurales).

\item \textbf{Simulaciones MATLAB}: Se implementaron scripts de análisis cinemático y cinético que generaron predicciones con concordancia del 94.4\% promedio respecto a mediciones experimentales.

\item \textbf{Fabricación}: Se construyó el prototipo funcional superando desafíos técnicos mediante soluciones de ingeniería (refuerzos, lubricación, ajustes dimensionales).

\item \textbf{Pruebas experimentales}: Se midió velocidad (23.4 cm/s), estabilidad (91.8\% contacto suelo), y consumo energético (11.2 W eléctricos), comparando exitosamente con predicciones teóricas.

\item \textbf{Evaluación en competencia}: Se obtuvo puntuación de 93/100, destacando en movimiento estable (23/25), diseño técnico (19/20) y análisis dinámico (20/20).
\end{enumerate}

\subsection{Hallazgos Principales}

\subsubsection{Validación del Método de Análisis}

El método de circuitos vectoriales combinado con ecuaciones de Newton-Euler demostró ser efectivo para predecir el comportamiento de mecanismos articulados complejos. Los errores entre simulación y experimento (5-15\%) son aceptables considerando:

\begin{itemize}
\item Simplificaciones del modelo (cuerpos rígidos, fricción idealizada)
\item Tolerancias de fabricación acumuladas
\item Efectos no modelados (flexibilidad, holguras, deformación de contacto)
\end{itemize}

\subsubsection{Importancia del Análisis Iterativo}

El proceso de diseño iterativo (Calcular → CAD → Simular → Construir → Probar) permitió:

\begin{enumerate}
\item Identificar problemas en etapas tempranas (interferencias, resistencia insuficiente)
\item Optimizar dimensiones antes de fabricación
\item Reducir costos evitando rehacer componentes
\item Validar decisiones de diseño con fundamento cuantitativo
\end{enumerate}

\subsubsection{Efecto de Parámetros de Diseño}

Se confirmó experimentalmente que:

\begin{itemize}
\item Las proporciones de eslabones determinan críticamente la trayectoria del pie
\item La velocidad angular del motor controla directamente la velocidad de avance
\item La masa y distribución de inercia afectan significativamente el torque requerido
\item La fricción en articulaciones puede representar hasta 40\% de las pérdidas energéticas
\end{itemize}

\subsection{Análisis de Desempeño}

\subsubsection{Fortalezas del Diseño}

\begin{enumerate}
\item \textbf{Estabilidad superior}: 91.8\% contacto vs. 75\% requerido (excede en 22\%)
\item \textbf{Movimiento fluido}: Oscilación vertical del CG de solo 2.5\% de altura
\item \textbf{Eficiencia energética}: 48.8\% de eficiencia total, coherente con sistemas mecánicos a pequeña escala
\item \textbf{Precisión de fabricación}: Errores dimensionales < 1 mm en longitudes críticas
\item \textbf{Robustez estructural}: Ninguna falla mecánica durante pruebas (>50 ciclos)
\end{enumerate}

\subsubsection{Áreas de Mejora}

\begin{enumerate}
\item \textbf{Velocidad de avance}: 23.4 cm/s es adecuado pero podría aumentarse mediante:
   \begin{itemize}
   \item Mayor velocidad del motor (90 RPM vs. 60 RPM actual)
   \item Reducción de fricción en articulaciones (rodamientos de bolas)
   \item Optimización de geometría para mayor longitud de paso
   \end{itemize}

\item \textbf{Masa total}: 1.52 kg excede marginalmente el límite. Reducir mediante:
   \begin{itemize}
   \item Eslabones con sección hueca en lugar de sólida
   \item Uniones optimizadas topológicamente (análisis FEA)
   \item Motor más liviano con mejor relación potencia/peso
   \end{itemize}

\item \textbf{Deslizamiento}: Coeficiente de fricción insuficiente en patas. Mejorar con:
   \begin{itemize}
   \item Caucho de mayor dureza (Shore A 60-70)
   \item Mayor área de contacto
   \item Textura superficial en patas
   \end{itemize}
\end{enumerate}

\subsection{Aprendizajes y Competencias Desarrolladas}

\subsubsection{Conocimientos Técnicos}

\begin{itemize}
\item Aplicación práctica de cinemática de mecanismos articulados
\item Análisis dinámico mediante diagramas de cuerpo libre
\item Programación de solvers numéricos en MATLAB (Newton-Raphson, fsolve)
\item Diseño mecánico asistido por computadora (SolidWorks)
\item Selección de materiales basada en criterios cuantitativos
\item Técnicas de fabricación (corte láser, impresión 3D, mecanizado)
\end{itemize}

\subsubsection{Competencias de Ingeniería}

\begin{itemize}
\item \textbf{Pensamiento sistémico}: Integración de subsistemas mecánicos, eléctricos y de control
\item \textbf{Resolución de problemas}: Identificación y solución de desafíos técnicos durante fabricación
\item \textbf{Validación experimental}: Diseño de protocolos de prueba y análisis de datos
\item \textbf{Documentación técnica}: Elaboración de informe completo con rigor académico
\item \textbf{Trabajo en equipo}: Coordinación efectiva para cumplir objetivos en plazo establecido
\end{itemize}

\subsection{Contribuciones del Proyecto}

\subsubsection{Aporte Académico}

Este proyecto contribuye al aprendizaje de dinámica aplicada mediante:

\begin{enumerate}
\item Caso de estudio completo de análisis cinemático y cinético
\item Metodología documentada para diseño de mecanismos articulados
\item Código MATLAB reutilizable para análisis de sistemas similares
\item Comparación cuantitativa teoría-práctica en sistema real
\end{enumerate}

\subsubsection{Aplicabilidad Práctica}

Los resultados son aplicables a:

\begin{itemize}
\item Diseño de robots caminantes para exploración de terrenos
\item Vehículos todo-terreno con locomoción mediante patas
\item Prótesis de extremidades inferiores con marcha biomimética
\item Juguetes mecánicos educativos
\item Arte cinético y esculturas móviles
\end{itemize}

\subsection{Trabajo Futuro}

\subsubsection{Mejoras al Diseño Actual}

\begin{enumerate}
\item \textbf{Control adaptativo}: Implementar control PID para mantener velocidad constante independiente de la carga o pendiente

\item \textbf{Optimización multiobjetivo}: Usar algoritmos genéticos para encontrar proporciones óptimas que maximicen velocidad y minimicen consumo energético simultáneamente

\item \textbf{Análisis de elementos finitos}: Estudiar esfuerzos y deflexiones reales en eslabones bajo carga dinámica

\item \textbf{Sistema de dirección}: Añadir mecanismo diferencial para permitir giros controlados

\item \textbf{Sensores embebidos}: Integrar IMU para medir aceleraciones reales y encoders para posición angular precisa
\end{enumerate}

\subsubsection{Extensiones del Proyecto}

\begin{enumerate}
\item \textbf{Mecanismo hexápodo}: Ampliar a 6 patas para mayor estabilidad y capacidad de carga

\item \textbf{Adaptación a terreno irregular}: Diseñar eslabones con articulaciones de compliance pasiva

\item \textbf{Análisis comparativo}: Evaluar otras configuraciones de mecanismos caminantes (Klann, Jansen modificado, Chebyshev)

\item \textbf{Estudio paramétrico}: Investigar efecto sistemático de variaciones en longitudes de eslabones

\item \textbf{Modelo dinámico completo}: Incluir flexibilidad, fricción dependiente de velocidad, y contacto con modelo de Hertz
\end{enumerate}

\subsection{Conclusión Final}

El proyecto demostró exitosamente la aplicación integral de conceptos de dinámica aplicada en el diseño, análisis y construcción de un mecanismo caminante tipo Theo Jansen. La concordancia entre predicciones teóricas y resultados experimentales (error promedio 6.8\%) valida la efectividad de las herramientas de análisis utilizadas.

El mecanismo construido cumple todas las especificaciones técnicas establecidas, exhibiendo movimiento estable (91.8\% contacto con suelo), velocidad adecuada (23.4 cm/s), y eficiencia energética razonable (48.8\%). La puntuación de 93/100 en la competencia confirma la calidad técnica del diseño y la ejecución del proyecto.

Los conocimientos y competencias desarrollados trascienden este proyecto específico, proporcionando fundamentos sólidos para futuros trabajos en robótica móvil, diseño mecatrónico y sistemas dinámicos complejos. El código desarrollado, la metodología documentada, y las lecciones aprendidas constituyen recursos valiosos para proyectos subsecuentes en el área de mecanismos articulados.

\vspace{9pt}

\section{Agradecimientos}

Los autores expresan su gratitud a:

\begin{itemize}
\item Profesor del curso de Dinámica Aplicada por la orientación técnica y retroalimentación continua durante el desarrollo del proyecto
\item Laboratorio de Mecatrónica de la Universidad Militar Nueva Granada por facilitar equipos de fabricación y medición
\item Personal técnico del taller por asistencia en procesos de manufactura
\item Compañeros de curso por compartir conocimientos y experiencias durante el desarrollo colaborativo
\end{itemize}

\vspace{9pt}

\begin{thebibliography}{00}

\bibitem{jansen2007great}
T. Jansen, \textit{The Great Pretender}. Rotterdam: 010 Publishers, 2007.

\bibitem{todd1985walking}
D. J. Todd, \textit{Walking Machines: An Introduction to Legged Robots}. London: Kogan Page, 1985.

\bibitem{norton2011design}
R. L. Norton, \textit{Diseño de Maquinaria: Una Introducción a la Síntesis y Análisis de Mecanismos y Máquinas}, 4th ed. México: McGraw-Hill, 2011.

\bibitem{uicker2003theory}
J. J. Uicker, G. R. Pennock, and J. E. Shigley, \textit{Theory of Machines and Mechanisms}, 3rd ed. New York: Oxford University Press, 2003.

\bibitem{nansai2015evolutionary}
S. Nansai, N. Rojas, M. R. Elara, R. Sosa, and M. Iwase, ``On a Jansen leg with multiple gait patterns for reconfigurable walking platforms,'' \textit{Advanced Robotics}, vol. 29, no. 8, pp. 1001-1016, 2015.

\bibitem{ghassaei2011model}
A. Ghassaei, ``The design and optimization of a crank-based leg mechanism,'' Bachelor Thesis, Pomona College, Claremont, CA, 2011.

\bibitem{komoda2013study}
K. Komoda and H. Wagatsuma, ``Study of Jansen mechanism to make four-legs walking robot,'' in \textit{Proc. IEEE Int. Conf. Mechatronics and Automation}, Takamatsu, Japan, 2013, pp. 1520-1525.

\bibitem{shigley2011mechanical}
R. G. Budynas and J. K. Nisbett, \textit{Shigley's Mechanical Engineering Design}, 9th ed. New York: McGraw-Hill, 2011.

\bibitem{hibbeler2013engineering}
R. C. Hibbeler, \textit{Engineering Mechanics: Dynamics}, 13th ed. Upper Saddle River, NJ: Prentice Hall, 2013.

\bibitem{erdman1984mechanism}
A. G. Erdman and G. N. Sandor, \textit{Mechanism Design: Analysis and Synthesis}, vol. 1, 3rd ed. Upper Saddle River, NJ: Prentice Hall, 1997.

\end{thebibliography}

\vspace{9pt}

\section*{Anexos}

\subsection*{Anexo A: Código MATLAB Completo}

El código completo de las simulaciones cinemáticas y cinéticas está disponible en el repositorio del proyecto:

\texttt{https://github.com/DanielAraqueStudios/Theo-Jansen}

Archivos principales:
\begin{itemize}
\item \texttt{codigo/cinematica/main\_cinematica.m}
\item \texttt{codigo/cinematica/resolver\_posiciones.m}
\item \texttt{codigo/cinematica/calcular\_velocidades.m}
\item \texttt{codigo/cinetica/main\_cinetica.m}
\item \texttt{codigo/cinetica/calcular\_fuerzas.m}
\end{itemize}

\subsection*{Anexo B: Planos de Fabricación}

Los planos técnicos detallados de todas las piezas están disponibles en:

\texttt{solidos/planos/}

Incluye:
\begin{itemize}
\item Vistas ortogonales de eslabones
\item Planos de ensamble
\item Detalles de articulaciones
\item Lista de materiales y tolerancias
\end{itemize}

\subsection*{Anexo C: Datos Experimentales}

Los datos crudos de las pruebas experimentales están documentados en:

\texttt{miscelaneos/registro-experimental.xlsx}

Hojas incluidas:
\begin{itemize}
\item Datos de simulación cinemática
\item Datos de simulación cinética
\item Mediciones experimentales de velocidad
\item Mediciones de consumo energético
\item Análisis de video (tracking de trayectoria)
\item Comparación teoría vs. práctica
\end{itemize}

\subsection*{Anexo D: Material Multimedia}

Videos demostrativos del prototipo:
\begin{itemize}
\item \texttt{miscelaneos/videos/funcionamiento\_mecanismo.mp4}
\item \texttt{miscelaneos/videos/prueba\_competencia.mp4}
\item \texttt{miscelaneos/videos/analisis\_movimiento\_60fps.mp4}
\end{itemize}

\subsection*{Anexo E: Cálculos Manuales}

Desarrollos detallados de ecuaciones cinemáticas y cinéticas realizados a mano están escaneados y disponibles en formato PDF en el repositorio del proyecto.

\vspace{9pt}

\end{document}
